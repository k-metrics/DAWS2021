% --- BXjsクラス用 ------------------------------------------------------------
% https://ctan.math.washington.edu/tex-archive/language/japanese/BX/bxjscls/bxjscls-manual.pdf
% --- 参考資料 ----------------------------------------------------------------
% https://github.com/Gedevan-Aleksizde/Japan.R2019/blob/master/latex/preamble.tex
% https://teastat.blogspot.com/2019/01/bookdown.html

% --- Document Class ----------------------------------------------------------
% Rmd側で指定した方が分かりやすいかも
% \documentclass[a4paper,xelatex,ja=standard]{bxjsarticle}

% --- Packages ----------------------------------------------------------------
% 日本語とkableExtraを使うために必要なTeXパッケージ指定
%  A4 210mm x 297mm
% \usepackage[a4paper]{geometry}         % Rmdのclassoptionでも指定可能
\usepackage{indentfirst}               % tinytexのリポジトリには存在しない?
\usepackage{booktabs}                  % ここからkableExtra用パッケージ
\usepackage{longtable}                 % 1 column modeのみで利用可
\usepackage{array}                     % 
\usepackage{multirow}                  % 
\usepackage{wrapfig}                   % 
\usepackage{float}                     % 
\usepackage{colortbl}                  % 
\usepackage{pdflscape}                 % 
\usepackage{tabu}                      % 
\usepackage{threeparttable}            % 
\usepackage{threeparttablex}           % 
\usepackage[normalem]{ulem}            % 
\usepackage{inputenc}                  % 
\usepackage{makecell}                  % 
\usepackage{xcolor}                    % ここまでkableExtra用
\usepackage{amsmath}                   % 
\usepackage{fontawesome5}              % fontawesomeを使うために必要
\usepackage{subfig}                    % 複数の図を並べる際に必要(古い?)
% \usepackage{subcaption}                % 同上(新しい?)
\usepackage{zxjatype}                  % 日本語処理に必要
% \usepackage{xeCJK}                     % zxjatypeを読み込むと自動で読み込む
\usepackage[noto]{zxjafont}            % Linux環境ではこちも使える
% \usepackage[haranoaji]{zxjafont}       % Windows環境ではこちらだけ
% \usepackage[hiragino-pro]{zxjafont}    % macOS環境用(おそらく、駄目ならNotoで)
\usepackage{pxrubrica}                 % ルビ用
\usepackage{hyperref}                  % ハイパーリンク用必要?

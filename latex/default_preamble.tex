% --- 参考資料 ----------------------------------------------------------------
% https://github.com/Gedevan-Aleksizde/Japan.R2019/blob/master/latex/preamble.tex
% https://teastat.blogspot.com/2019/01/bookdown.html

% --- Packages ----------------------------------------------------------------
% 日本語とtufte, kableExtraを使うために必要なTeXパッケージ指定
%  A4 210mm x 297mm
\usepackage[a4paper]{geometry}
\usepackage{indentfirst}                % tinytexのリポジトリには存在しない?
\usepackage{booktabs}                   % ここからkableExtra用パッケージ
\usepackage{longtable}                  % 
\usepackage{array}                      % 
\usepackage{multirow}                   % 
\usepackage{wrapfig}                    % 
\usepackage{float}                      % 
\usepackage{colortbl}                   % 
\usepackage{pdflscape}                  % 
\usepackage{tabu}                       % 
\usepackage{threeparttable}             % 
\usepackage{threeparttablex}            % 
\usepackage[normalem]{ulem}             % 
\usepackage{inputenc}                   % 
\usepackage{makecell}                   % 
\usepackage{xcolor}                     % ここまでkableExtra用
\usepackage{amsmath}                    % 
\usepackage{fontawesome5}               % fontawesomeを使うために必要
\usepackage{subfig}                     % 複数の図を並べる際に必要(古い?)
% \usepackage{subcaption}                 % 同上(新しい?)
\usepackage{xeCJK}                      % 以下、日本語フォント用に必要
% \usepackage[noto]{zxjafont}             % Linux環境ではこちも使える
\usepackage[haranoaji]{zxjafont}        % Windows環境ではこちらだけ
\usepackage{zxjatype}                   % 日本語処理に必要
\usepackage{pxrubrica}                  % ルビ用
\usepackage{hyperref}                   % ハイパーリンク用必要?

% ## Windows環境でNotoフォントを指定したい場合は以下のようにheader-includeで
% ## 個別に指定する(setCJKxxxfotnの指定は必要?)
% # \setmainfont{NotoSerifCJKjp-Regular.otf}[BoldFont=NotoSerifCJKjp-Bold.otf]
% # \setsansfont{NotoSansCJKjp-Regular.otf}[BoldFont=NotoSansCJKjp-Bold.otf]
% # \setmonofont{NotoSansMonoCJKjp-Regular.otf}[BoldFont=NotoSansMonoCJKjp-Bold.otf]
% ## モノフォントは源ノ角コード(Source Code Proの日本語版)がおすゝめ
% \setmonofont{SourceHanCodeJP-Regular.otf}[BoldFont=SourceHanCodeJPS-Bold.otf]

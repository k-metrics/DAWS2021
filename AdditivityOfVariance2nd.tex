\documentclass[a4paper]{tufte-handout}

% ams
\usepackage{amssymb,amsmath}

\usepackage{ifxetex,ifluatex}
\usepackage{fixltx2e} % provides \textsubscript
\ifnum 0\ifxetex 1\fi\ifluatex 1\fi=0 % if pdftex
  \usepackage[T1]{fontenc}
  \usepackage[utf8]{inputenc}
\else % if luatex or xelatex
  \makeatletter
  \@ifpackageloaded{fontspec}{}{\usepackage{fontspec}}
  \makeatother
  \defaultfontfeatures{Ligatures=TeX,Scale=MatchLowercase}
  \makeatletter
  \@ifpackageloaded{soul}{
     \renewcommand\allcapsspacing[1]{{\addfontfeature{LetterSpace=15}#1}}
     \renewcommand\smallcapsspacing[1]{{\addfontfeature{LetterSpace=10}#1}}
   }{}
  \makeatother

\fi

% graphix
\usepackage{graphicx}
\setkeys{Gin}{width=\linewidth,totalheight=\textheight,keepaspectratio}

% booktabs
\usepackage{booktabs}

% url
\usepackage{url}

% hyperref
\usepackage{hyperref}

% units.
\usepackage{units}


\setcounter{secnumdepth}{-1}

% citations
\usepackage{natbib}
\bibliographystyle{plainnat}


% pandoc syntax highlighting
\usepackage{color}
\usepackage{fancyvrb}
\newcommand{\VerbBar}{|}
\newcommand{\VERB}{\Verb[commandchars=\\\{\}]}
\DefineVerbatimEnvironment{Highlighting}{Verbatim}{commandchars=\\\{\}}
% Add ',fontsize=\small' for more characters per line
\newenvironment{Shaded}{}{}
\newcommand{\AlertTok}[1]{\textcolor[rgb]{1.00,0.00,0.00}{\textbf{#1}}}
\newcommand{\AnnotationTok}[1]{\textcolor[rgb]{0.38,0.63,0.69}{\textbf{\textit{#1}}}}
\newcommand{\AttributeTok}[1]{\textcolor[rgb]{0.49,0.56,0.16}{#1}}
\newcommand{\BaseNTok}[1]{\textcolor[rgb]{0.25,0.63,0.44}{#1}}
\newcommand{\BuiltInTok}[1]{#1}
\newcommand{\CharTok}[1]{\textcolor[rgb]{0.25,0.44,0.63}{#1}}
\newcommand{\CommentTok}[1]{\textcolor[rgb]{0.38,0.63,0.69}{\textit{#1}}}
\newcommand{\CommentVarTok}[1]{\textcolor[rgb]{0.38,0.63,0.69}{\textbf{\textit{#1}}}}
\newcommand{\ConstantTok}[1]{\textcolor[rgb]{0.53,0.00,0.00}{#1}}
\newcommand{\ControlFlowTok}[1]{\textcolor[rgb]{0.00,0.44,0.13}{\textbf{#1}}}
\newcommand{\DataTypeTok}[1]{\textcolor[rgb]{0.56,0.13,0.00}{#1}}
\newcommand{\DecValTok}[1]{\textcolor[rgb]{0.25,0.63,0.44}{#1}}
\newcommand{\DocumentationTok}[1]{\textcolor[rgb]{0.73,0.13,0.13}{\textit{#1}}}
\newcommand{\ErrorTok}[1]{\textcolor[rgb]{1.00,0.00,0.00}{\textbf{#1}}}
\newcommand{\ExtensionTok}[1]{#1}
\newcommand{\FloatTok}[1]{\textcolor[rgb]{0.25,0.63,0.44}{#1}}
\newcommand{\FunctionTok}[1]{\textcolor[rgb]{0.02,0.16,0.49}{#1}}
\newcommand{\ImportTok}[1]{#1}
\newcommand{\InformationTok}[1]{\textcolor[rgb]{0.38,0.63,0.69}{\textbf{\textit{#1}}}}
\newcommand{\KeywordTok}[1]{\textcolor[rgb]{0.00,0.44,0.13}{\textbf{#1}}}
\newcommand{\NormalTok}[1]{#1}
\newcommand{\OperatorTok}[1]{\textcolor[rgb]{0.40,0.40,0.40}{#1}}
\newcommand{\OtherTok}[1]{\textcolor[rgb]{0.00,0.44,0.13}{#1}}
\newcommand{\PreprocessorTok}[1]{\textcolor[rgb]{0.74,0.48,0.00}{#1}}
\newcommand{\RegionMarkerTok}[1]{#1}
\newcommand{\SpecialCharTok}[1]{\textcolor[rgb]{0.25,0.44,0.63}{#1}}
\newcommand{\SpecialStringTok}[1]{\textcolor[rgb]{0.73,0.40,0.53}{#1}}
\newcommand{\StringTok}[1]{\textcolor[rgb]{0.25,0.44,0.63}{#1}}
\newcommand{\VariableTok}[1]{\textcolor[rgb]{0.10,0.09,0.49}{#1}}
\newcommand{\VerbatimStringTok}[1]{\textcolor[rgb]{0.25,0.44,0.63}{#1}}
\newcommand{\WarningTok}[1]{\textcolor[rgb]{0.38,0.63,0.69}{\textbf{\textit{#1}}}}

% longtable
\usepackage{longtable,booktabs}

% multiplecol
\usepackage{multicol}

% strikeout
\usepackage[normalem]{ulem}

% morefloats
\usepackage{morefloats}


% tightlist macro required by pandoc >= 1.14
\providecommand{\tightlist}{%
  \setlength{\itemsep}{0pt}\setlength{\parskip}{0pt}}

% title / author / date
\title[分散の加法性を視覚的に理解する(その2)]{分散の加法性を視覚的に理解する(その2)}
\author{Sampo Suzuki, CC 4.0 BY-NC-SA}
\date{2021-06-01}

% --- 参考資料 ----------------------------------------------------------------
% https://github.com/Gedevan-Aleksizde/Japan.R2019/blob/master/latex/preamble.tex
% https://teastat.blogspot.com/2019/01/bookdown.html

% --- Packages ----------------------------------------------------------------
% 日本語とtufte, kableExtraを使うために必要なTeXパッケージ指定
% tufteではA4サイズの指定が不可能
%  A4 210mm x 297mm
%   \usepackage[a4paper, total={6.5in, 9.5in}]{geometry}
%   \usepackage{indentfirst}   # tinytexのリポジトリには存在しない?
% \usepackage[a4paper, total={160mm, 247mm}, left=25mm, top=25mm]{geometry}
% \usepackage[pdfbox,tombo]{gentombow}  % トンボを設定する場合は有効にする
\usepackage{ifthen}                     % 条件分岐用 \ifthenelse{条件}{T}{F}
\usepackage{booktabs}                   % ここからkableExtra用パッケージ
\usepackage{longtable}                  % 
\usepackage{array}                      % 
\usepackage{multirow}                   % 
\usepackage{wrapfig}                    % 
\usepackage{float}                      % 
\usepackage{colortbl}                   % 
\usepackage{pdflscape}                  % 
\usepackage{tabu}                       % 
\usepackage{threeparttable}             % 
\usepackage{threeparttablex}            % 
\usepackage[normalem]{ulem}             % 
\usepackage{inputenc}                   % 
\usepackage{makecell}                   % 
\usepackage{xcolor}                     % ここまでkableExtra用
\usepackage{amsmath}                    % 
\usepackage{fontawesome5}               % fontawesomeを使うために必要
\usepackage{subfig}                     % 複数の図を並べる際に必要(古い?)
% \usepackage{subcaption}                 % 同上(新しい?)
\usepackage{xeCJK}                      % 以下、日本語フォント用に必要
\usepackage[noto]{zxjafont}             % Linux環境ではこちを指定
% \usepackage[haranoaji]{zxjafont}      % Windows環境ではこちらを指定する
\usepackage{zxjatype}                   % 日本語処理に必要
\usepackage{pxrubrica}                  % ルビ用
\usepackage{hyperref}                   % ハイパーリンク用必要?

% --- Index ------------------------------------------------------------------
% https://texwiki.texjp.org/?%E7%B4%A2%E5%BC%95%E4%BD%9C%E6%88%90
% これを指定するとIndex(索引)は作成されるが参照ページがズレる
% 中間ファイルの.indではページはズレていないので、その後の結合処理がおかしい
% \usepackage{makeidx}
% \makeindex
% \usepackage{showidx}                  % 索引確認用

% --- Table of Contentes ------------------------------------------------------
% TOCにLOT(List of Tables), LOF(List of Figures), Bibliography, Indexを表示
% \usepackage[nottoc]{tocbibind}

% --- Fonts -------------------------------------------------------------------
% フォントしては index.html でも可能(pandoc用オプションは index.htmlにて)
% \setCJKmonofont{Source Han Code JP}
\setmonofont{Source Han Code JP}     % Linuxではこれのみコメントアウトする
% \setjamonofont{Source Han Code JP}

% ## 日本語フォントの扱いについてはzxjafontパッケージの解説を参照のこと
% # https://mirror.las.iastate.edu/tex-archive/language/japanese/zxjafont/zxjafont.pdf
% #
% ## Windows環境ではなぜかNotoフォントが認識されないので源ノシリーズベースの
% ## 原ノ味フォントかIPAexフォントを利用する(原ノ味はtlmgrでインストール可)
% # \usepackage[haranoaji]{zxjafont}
% # \usepackage[ipaex]{zxjafont}
% #
% ## Windows環境でNotoフォントを指定したい場合は以下のようにheader-includeで
% ## 個別に指定する(setCJKxxxfotnの指定は必要?)
% # \setmainfont{NotoSerifCJKjp-Regular.otf}[BoldFont=NotoSerifCJKjp-Bold.otf]
% # \setsansfont{NotoSansCJKjp-Regular.otf}[BoldFont=NotoSansCJKjp-Bold.otf]
% # \setmonofont{NotoSansMonoCJKjp-Regular.otf}[BoldFont=NotoSansMonoCJKjp-Bold.otf]
% ## モノフォントは源ノ角コード(Source Code Proの日本語版)がおすゝめ
% # \setmonofont{SourceHanCodeJP-Regular.otf}[BoldFont=SourceHanCodeJPS-Bold.otf]
\usepackage{booktabs}
\usepackage{longtable}
\usepackage{array}
\usepackage{multirow}
\usepackage{wrapfig}
\usepackage{float}
\usepackage{colortbl}
\usepackage{pdflscape}
\usepackage{tabu}
\usepackage{threeparttable}
\usepackage{threeparttablex}
\usepackage[normalem]{ulem}
\usepackage{makecell}
\usepackage{xcolor}

\begin{document}

\maketitle




\hypertarget{ux306fux3058ux3081ux306bux5c0fux5ba4ux5148ux751fux306eux30a2ux30c9ux30d0ux30a4ux30b9ux304bux3089}{%
\section{\texorpdfstring{\textbf{はじめに}(小室先生のアドバイスから)}{はじめに(小室先生のアドバイスから)}}\label{ux306fux3058ux3081ux306bux5c0fux5ba4ux5148ux751fux306eux30a2ux30c9ux30d0ux30a4ux30b9ux304bux3089}}

 分散の加法性が成り立つには「データが独立」であるという前提条件があります。乱数生成した二つのデータ\footnote{各々\texttt{rr\ n}個のデータ}が本当に独立なのかを確認すると共に分散の加法性も確認してみます。

 

\hypertarget{ux95a2ux6570ux306eux5b9aux7fa9}{%
\subsection{\texorpdfstring{\textbf{関数の定義}}{関数の定義}}\label{ux95a2ux6570ux306eux5b9aux7fa9}}

 最初に以下の処理を行う関数を定義します。

\begin{itemize}
\tightlist
\item
  データを乱数生成する\footnote{今回は\texttt{rnorm()}関数による分散が\(100\)となる正規分布}
\item
  乱数生成したデータをランダムサンプリングする\footnote{\texttt{sampling\ =\ TRUE}の場合のみ}
\item
  作成したデータの統計量を求める
\item
  無相関検定の結果と統計量をデータフレームにまとめる
\end{itemize}

\begin{Shaded}
\begin{Highlighting}[numbers=left,,]
\NormalTok{f }\OtherTok{\textless{}{-}} \ControlFlowTok{function}\NormalTok{(}\AttributeTok{i =} \ConstantTok{NA}\NormalTok{, }\AttributeTok{sampling =} \ConstantTok{FALSE}\NormalTok{, }\AttributeTok{n =} \DecValTok{5000000}\NormalTok{) \{}
  \CommentTok{\# データを乱数生成する}
\NormalTok{  x }\OtherTok{\textless{}{-}} \FunctionTok{rnorm}\NormalTok{(}\AttributeTok{n =}\NormalTok{ n, }\AttributeTok{mean =} \DecValTok{10}\NormalTok{, }\AttributeTok{sd =} \DecValTok{10}\NormalTok{)}
\NormalTok{  y }\OtherTok{\textless{}{-}} \FunctionTok{rnorm}\NormalTok{(}\AttributeTok{n =}\NormalTok{ n, }\AttributeTok{mean =} \DecValTok{30}\NormalTok{, }\AttributeTok{sd =} \DecValTok{10}\NormalTok{)}
  \CommentTok{\# 乱数生成したデータからサンプリングする場合}
  \ControlFlowTok{if}\NormalTok{ (sampling }\SpecialCharTok{==} \ConstantTok{TRUE}\NormalTok{) \{}
\NormalTok{    x }\OtherTok{\textless{}{-}} \FunctionTok{sample}\NormalTok{(x, n, }\AttributeTok{replace =} \ConstantTok{TRUE}\NormalTok{)}
\NormalTok{    y }\OtherTok{\textless{}{-}} \FunctionTok{sample}\NormalTok{(y, n, }\AttributeTok{replace =} \ConstantTok{TRUE}\NormalTok{)}
\NormalTok{  \}}
  \CommentTok{\# 統計量を求める}
\NormalTok{  df }\OtherTok{\textless{}{-}} \FunctionTok{data.frame}\NormalTok{(}\AttributeTok{no =}\NormalTok{ i, }\AttributeTok{var.x =} \FunctionTok{var}\NormalTok{(x), }\AttributeTok{var.y =} \FunctionTok{var}\NormalTok{(y),}
                   \AttributeTok{var.xy =} \FunctionTok{var}\NormalTok{(x }\SpecialCharTok{+}\NormalTok{ y), }\AttributeTok{var.sum =} \FunctionTok{var}\NormalTok{(x) }\SpecialCharTok{+} \FunctionTok{var}\NormalTok{(y),}
                   \AttributeTok{cov =} \FunctionTok{cov}\NormalTok{(x, y), }\AttributeTok{cov2 =} \FunctionTok{cov}\NormalTok{(x, y) }\SpecialCharTok{*} \DecValTok{2}\NormalTok{)}
  \CommentTok{\# 無相関の検定結果と統計量をデータフレームにまとめる}
\NormalTok{  df }\OtherTok{\textless{}{-}} \FunctionTok{cor.test}\NormalTok{(x, y) }\SpecialCharTok{\%\textgreater{}\%}\NormalTok{ broom}\SpecialCharTok{::}\FunctionTok{tidy}\NormalTok{() }\SpecialCharTok{\%\textgreater{}\%}\NormalTok{ dplyr}\SpecialCharTok{::}\FunctionTok{bind\_cols}\NormalTok{(df)}
  \FunctionTok{return}\NormalTok{(df)}
\NormalTok{\}}
\end{Highlighting}
\end{Shaded}

この関数を\texttt{for}ループで一定回数繰り返し、その結果をデータフレームにまとめ、分散がどのようになっているかを比較します。

\newpage

\begin{longtable}[]{@{}lll@{}}
\caption{変数の意味}\tabularnewline
\toprule
変数名 & その意味 & 備考 \\
\midrule
\endfirsthead
\toprule
変数名 & その意味 & 備考 \\
\midrule
\endhead
\texttt{var.x} & データ\texttt{x}の分散 & \\
\texttt{var.y} & データ\texttt{y}の分散 & \\
\texttt{var.xy} &
データ\texttt{x}と\texttt{y}を加算したものの分散(\texttt{var(x\ +\ y)})
& 加法1 \\
\texttt{var.sum} & データ\texttt{x},
\texttt{y}の分散を加算したもの(\texttt{var(x)\ +\ var(y)}) & 加法2 \\
\texttt{var.diff} & \texttt{var.xy}から\texttt{var.sum}を減算したもの &
加法1と加法2の差異 \\
\texttt{cov2} & データ\texttt{x}, \texttt{y}の共分散の2倍数 & \\
\texttt{cov} & データ\texttt{x}, \texttt{y}の共分散 &
計算のみで未出力 \\
\bottomrule
\end{longtable}

\newpage

\hypertarget{ux4e71ux6570ux751fux6210ux3057ux305fux30c7ux30fcux30bfux306eux5834ux5408}{%
\subsection{\texorpdfstring{\textbf{乱数生成したデータの場合}}{乱数生成したデータの場合}}\label{ux4e71ux6570ux751fux6210ux3057ux305fux30c7ux30fcux30bfux306eux5834ux5408}}

\begin{table}

\caption{\label{tab:unnamed-chunk-3}乱数生成した二つのデータの分散}
\centering
\resizebox{\linewidth}{!}{
\begin{tabular}[t]{rrrrrrrrr}
\toprule
No & 相関係数 & p値 & 標本x & 標本y & 加法1 & 加法2 & 差異 & cov2\\
\midrule
\cellcolor{gray!6}{1} & \cellcolor{gray!6}{-0.0001291} & \cellcolor{gray!6}{0.7728236} & \cellcolor{gray!6}{99.94993} & \cellcolor{gray!6}{100.02913} & \cellcolor{gray!6}{199.9532} & \cellcolor{gray!6}{199.9791} & \cellcolor{gray!6}{-0.0258179} & \cellcolor{gray!6}{-0.0258179}\\
2 & 0.0003030 & 0.4981275 & 100.03307 & 100.02382 & 200.1175 & 200.0569 & 0.0606092 & 0.0606092\\
\cellcolor{gray!6}{3} & \cellcolor{gray!6}{-0.0004276} & \cellcolor{gray!6}{0.3389610} & \cellcolor{gray!6}{99.99416} & \cellcolor{gray!6}{100.00821} & \cellcolor{gray!6}{199.9168} & \cellcolor{gray!6}{200.0024} & \cellcolor{gray!6}{-0.0855280} & \cellcolor{gray!6}{-0.0855280}\\
4 & 0.0004838 & 0.2793815 & 100.00953 & 99.96570 & 200.0720 & 199.9752 & 0.0967391 & 0.0967391\\
\cellcolor{gray!6}{5} & \cellcolor{gray!6}{-0.0002187} & \cellcolor{gray!6}{0.6247927} & \cellcolor{gray!6}{100.06228} & \cellcolor{gray!6}{100.07481} & \cellcolor{gray!6}{200.0933} & \cellcolor{gray!6}{200.1371} & \cellcolor{gray!6}{-0.0437737} & \cellcolor{gray!6}{-0.0437737}\\
\addlinespace
6 & 0.0001338 & 0.7647540 & 100.07596 & 99.99899 & 200.1017 & 200.0749 & 0.0267752 & 0.0267752\\
\cellcolor{gray!6}{7} & \cellcolor{gray!6}{0.0005169} & \cellcolor{gray!6}{0.2477801} & \cellcolor{gray!6}{99.98430} & \cellcolor{gray!6}{99.88829} & \cellcolor{gray!6}{199.9759} & \cellcolor{gray!6}{199.8726} & \cellcolor{gray!6}{0.1033083} & \cellcolor{gray!6}{0.1033083}\\
9 & 0.0005154 & 0.2491533 & 99.98472 & 99.96028 & 200.0481 & 199.9450 & 0.1030462 & 0.1030462\\
\cellcolor{gray!6}{10} & \cellcolor{gray!6}{0.0002321} & \cellcolor{gray!6}{0.6038112} & \cellcolor{gray!6}{99.94002} & \cellcolor{gray!6}{99.94139} & \cellcolor{gray!6}{199.9278} & \cellcolor{gray!6}{199.8814} & \cellcolor{gray!6}{0.0463868} & \cellcolor{gray!6}{0.0463868}\\
11 & 0.0005929 & 0.1848895 & 99.99015 & 99.99617 & 200.1049 & 199.9863 & 0.1185795 & 0.1185795\\
\addlinespace
\cellcolor{gray!6}{12} & \cellcolor{gray!6}{-0.0006464} & \cellcolor{gray!6}{0.1483490} & \cellcolor{gray!6}{99.99653} & \cellcolor{gray!6}{99.94682} & \cellcolor{gray!6}{199.8141} & \cellcolor{gray!6}{199.9433} & \cellcolor{gray!6}{-0.1292428} & \cellcolor{gray!6}{-0.1292428}\\
13 & -0.0002263 & 0.6127855 & 99.95193 & 100.09496 & 200.0016 & 200.0469 & -0.0452777 & -0.0452777\\
\cellcolor{gray!6}{14} & \cellcolor{gray!6}{0.0001560} & \cellcolor{gray!6}{0.7272108} & \cellcolor{gray!6}{100.13666} & \cellcolor{gray!6}{99.89418} & \cellcolor{gray!6}{200.0620} & \cellcolor{gray!6}{200.0308} & \cellcolor{gray!6}{0.0312058} & \cellcolor{gray!6}{0.0312058}\\
15 & 0.0006594 & 0.1403860 & 100.03624 & 99.94274 & 200.1108 & 199.9790 & 0.1318565 & 0.1318565\\
\cellcolor{gray!6}{16} & \cellcolor{gray!6}{0.0004778} & \cellcolor{gray!6}{0.2853648} & \cellcolor{gray!6}{99.92304} & \cellcolor{gray!6}{100.07177} & \cellcolor{gray!6}{200.0904} & \cellcolor{gray!6}{199.9948} & \cellcolor{gray!6}{0.0955532} & \cellcolor{gray!6}{0.0955532}\\
\addlinespace
17 & -0.0003282 & 0.4630601 & 100.01961 & 100.06666 & 200.0206 & 200.0863 & -0.0656629 & -0.0656629\\
\cellcolor{gray!6}{18} & \cellcolor{gray!6}{-0.0001478} & \cellcolor{gray!6}{0.7410526} & \cellcolor{gray!6}{99.96112} & \cellcolor{gray!6}{100.17508} & \cellcolor{gray!6}{200.1066} & \cellcolor{gray!6}{200.1362} & \cellcolor{gray!6}{-0.0295773} & \cellcolor{gray!6}{-0.0295773}\\
19 & -0.0006194 & 0.1660747 & 99.99906 & 99.94166 & 199.8169 & 199.9407 & -0.1238349 & -0.1238349\\
\cellcolor{gray!6}{20} & \cellcolor{gray!6}{-0.0006640} & \cellcolor{gray!6}{0.1376396} & \cellcolor{gray!6}{99.97401} & \cellcolor{gray!6}{100.05617} & \cellcolor{gray!6}{199.8974} & \cellcolor{gray!6}{200.0302} & \cellcolor{gray!6}{-0.1328101} & \cellcolor{gray!6}{-0.1328101}\\
21 & 0.0003286 & 0.4624798 & 99.90516 & 99.98661 & 199.9575 & 199.8918 & 0.0656842 & 0.0656842\\
\addlinespace
\cellcolor{gray!6}{22} & \cellcolor{gray!6}{0.0003411} & \cellcolor{gray!6}{0.4455869} & \cellcolor{gray!6}{100.03523} & \cellcolor{gray!6}{99.96079} & \cellcolor{gray!6}{200.0642} & \cellcolor{gray!6}{199.9960} & \cellcolor{gray!6}{0.0682249} & \cellcolor{gray!6}{0.0682249}\\
23 & -0.0000975 & 0.8274028 & 99.93787 & 99.95452 & 199.8729 & 199.8924 & -0.0194910 & -0.0194910\\
\cellcolor{gray!6}{25} & \cellcolor{gray!6}{0.0002461} & \cellcolor{gray!6}{0.5820510} & \cellcolor{gray!6}{99.98915} & \cellcolor{gray!6}{100.07462} & \cellcolor{gray!6}{200.1130} & \cellcolor{gray!6}{200.0638} & \cellcolor{gray!6}{0.0492442} & \cellcolor{gray!6}{0.0492442}\\
26 & 0.0002765 & 0.5364458 & 100.03539 & 99.96041 & 200.0511 & 199.9958 & 0.0552920 & 0.0552920\\
\cellcolor{gray!6}{27} & \cellcolor{gray!6}{0.0000876} & \cellcolor{gray!6}{0.8447762} & \cellcolor{gray!6}{99.95505} & \cellcolor{gray!6}{99.98080} & \cellcolor{gray!6}{199.9534} & \cellcolor{gray!6}{199.9358} & \cellcolor{gray!6}{0.0175062} & \cellcolor{gray!6}{0.0175062}\\
\addlinespace
28 & 0.0003500 & 0.4338142 & 100.03738 & 99.97071 & 200.0781 & 200.0081 & 0.0700080 & 0.0700080\\
\cellcolor{gray!6}{29} & \cellcolor{gray!6}{-0.0002572} & \cellcolor{gray!6}{0.5652014} & \cellcolor{gray!6}{99.96146} & \cellcolor{gray!6}{99.97635} & \cellcolor{gray!6}{199.8864} & \cellcolor{gray!6}{199.9378} & \cellcolor{gray!6}{-0.0514255} & \cellcolor{gray!6}{-0.0514255}\\
30 & 0.0000473 & 0.9157339 & 99.95647 & 99.97285 & 199.9388 & 199.9293 & 0.0094605 & 0.0094605\\
\cellcolor{gray!6}{31} & \cellcolor{gray!6}{0.0000813} & \cellcolor{gray!6}{0.8557782} & \cellcolor{gray!6}{100.03076} & \cellcolor{gray!6}{99.98796} & \cellcolor{gray!6}{200.0350} & \cellcolor{gray!6}{200.0187} & \cellcolor{gray!6}{0.0162578} & \cellcolor{gray!6}{0.0162578}\\
32 & 0.0002302 & 0.6067843 & 100.04779 & 99.98003 & 200.0739 & 200.0278 & 0.0460398 & 0.0460398\\
\addlinespace
\cellcolor{gray!6}{33} & \cellcolor{gray!6}{0.0004089} & \cellcolor{gray!6}{0.3604991} & \cellcolor{gray!6}{100.03151} & \cellcolor{gray!6}{100.06965} & \cellcolor{gray!6}{200.1830} & \cellcolor{gray!6}{200.1012} & \cellcolor{gray!6}{0.0818291} & \cellcolor{gray!6}{0.0818291}\\
34 & 0.0006007 & 0.1791799 & 100.13358 & 100.03684 & 200.2907 & 200.1704 & 0.1202494 & 0.1202494\\
\cellcolor{gray!6}{35} & \cellcolor{gray!6}{0.0001666} & \cellcolor{gray!6}{0.7094596} & \cellcolor{gray!6}{100.05569} & \cellcolor{gray!6}{100.06161} & \cellcolor{gray!6}{200.1507} & \cellcolor{gray!6}{200.1173} & \cellcolor{gray!6}{0.0333443} & \cellcolor{gray!6}{0.0333443}\\
36 & -0.0008561 & 0.0555704 & 100.08053 & 100.04790 & 199.9571 & 200.1284 & -0.1713382 & -0.1713382\\
\cellcolor{gray!6}{37} & \cellcolor{gray!6}{-0.0000164} & \cellcolor{gray!6}{0.9707969} & \cellcolor{gray!6}{100.00381} & \cellcolor{gray!6}{100.01701} & \cellcolor{gray!6}{200.0175} & \cellcolor{gray!6}{200.0208} & \cellcolor{gray!6}{-0.0032747} & \cellcolor{gray!6}{-0.0032747}\\
\addlinespace
38 & -0.0000898 & 0.8408415 & 100.17939 & 99.92802 & 200.0894 & 200.1074 & -0.0179713 & -0.0179713\\
\cellcolor{gray!6}{39} & \cellcolor{gray!6}{-0.0001503} & \cellcolor{gray!6}{0.7367628} & \cellcolor{gray!6}{100.00455} & \cellcolor{gray!6}{100.01916} & \cellcolor{gray!6}{199.9936} & \cellcolor{gray!6}{200.0237} & \cellcolor{gray!6}{-0.0300691} & \cellcolor{gray!6}{-0.0300691}\\
40 & -0.0000712 & 0.8734625 & 99.97557 & 100.07320 & 200.0345 & 200.0488 & -0.0142483 & -0.0142483\\
\cellcolor{gray!6}{41} & \cellcolor{gray!6}{0.0000246} & \cellcolor{gray!6}{0.9560569} & \cellcolor{gray!6}{100.00854} & \cellcolor{gray!6}{100.02151} & \cellcolor{gray!6}{200.0350} & \cellcolor{gray!6}{200.0300} & \cellcolor{gray!6}{0.0049292} & \cellcolor{gray!6}{0.0049292}\\
42 & 0.0004766 & 0.2865886 & 100.10682 & 100.07505 & 200.2773 & 200.1819 & 0.0954000 & 0.0954000\\
\addlinespace
\cellcolor{gray!6}{43} & \cellcolor{gray!6}{-0.0000857} & \cellcolor{gray!6}{0.8479615} & \cellcolor{gray!6}{100.00054} & \cellcolor{gray!6}{100.02174} & \cellcolor{gray!6}{200.0051} & \cellcolor{gray!6}{200.0223} & \cellcolor{gray!6}{-0.0171499} & \cellcolor{gray!6}{-0.0171499}\\
44 & -0.0004231 & 0.3440610 & 100.09685 & 100.00361 & 200.0158 & 200.1005 & -0.0846707 & -0.0846707\\
\cellcolor{gray!6}{45} & \cellcolor{gray!6}{-0.0006599} & \cellcolor{gray!6}{0.1400776} & \cellcolor{gray!6}{99.96225} & \cellcolor{gray!6}{100.01485} & \cellcolor{gray!6}{199.8451} & \cellcolor{gray!6}{199.9771} & \cellcolor{gray!6}{-0.1319578} & \cellcolor{gray!6}{-0.1319578}\\
\bottomrule
\end{tabular}}
\end{table}

\begin{table}

\caption{\label{tab:unnamed-chunk-3}乱数生成した二つのデータが独立でない場合}
\centering
\resizebox{\linewidth}{!}{
\begin{tabular}[t]{rrrrrrrrr}
\toprule
No & 相関係数 & p値 & 標本x & 標本y & 加法1 & 加法2 & 差異 & cov2\\
\midrule
\cellcolor{gray!6}{8} & \cellcolor{gray!6}{-0.0009668} & \cellcolor{gray!6}{0.0306226} & \cellcolor{gray!6}{100.0167} & \cellcolor{gray!6}{99.96575} & \cellcolor{gray!6}{199.7891} & \cellcolor{gray!6}{199.9825} & \cellcolor{gray!6}{-0.1933531} & \cellcolor{gray!6}{-0.1933531}\\
24 & -0.0010322 & 0.0209940 & 100.1072 & 100.03828 & 199.9389 & 200.1455 & -0.2065922 & -0.2065922\\
\bottomrule
\end{tabular}}
\end{table}

\[\mbox{加法1} = var(x + y), \mbox{加法2} = var(x) + var(y)\]

\newpage

\hypertarget{ux4e71ux6570ux751fux6210ux3057ux305fux30c7ux30fcux30bfux3092ux30e9ux30f3ux30c0ux30e0ux30b5ux30f3ux30d7ux30eaux30f3ux30b0ux3057ux305fux5834ux5408}{%
\subsection{\texorpdfstring{\textbf{乱数生成したデータをランダムサンプリングした場合}}{乱数生成したデータをランダムサンプリングした場合}}\label{ux4e71ux6570ux751fux6210ux3057ux305fux30c7ux30fcux30bfux3092ux30e9ux30f3ux30c0ux30e0ux30b5ux30f3ux30d7ux30eaux30f3ux30b0ux3057ux305fux5834ux5408}}

\begin{table}

\caption{\label{tab:unnamed-chunk-5}ランダムサンプリングしたデータの分散}
\centering
\resizebox{\linewidth}{!}{
\begin{tabular}[t]{rrrrrrrrr}
\toprule
No & 相関係数 & p値 & 標本x & 標本y & 加法1 & 加法2 & 差異 & cov2\\
\midrule
\cellcolor{gray!6}{1} & \cellcolor{gray!6}{-0.0000485} & \cellcolor{gray!6}{0.9135710} & \cellcolor{gray!6}{100.04656} & \cellcolor{gray!6}{99.92857} & \cellcolor{gray!6}{199.9654} & \cellcolor{gray!6}{199.9751} & \cellcolor{gray!6}{-0.0097065} & \cellcolor{gray!6}{-0.0097065}\\
2 & -0.0000652 & 0.8841659 & 99.98215 & 99.99819 & 199.9673 & 199.9803 & -0.0130297 & -0.0130297\\
\cellcolor{gray!6}{3} & \cellcolor{gray!6}{-0.0002734} & \cellcolor{gray!6}{0.5409125} & \cellcolor{gray!6}{100.04045} & \cellcolor{gray!6}{99.94448} & \cellcolor{gray!6}{199.9302} & \cellcolor{gray!6}{199.9849} & \cellcolor{gray!6}{-0.0546842} & \cellcolor{gray!6}{-0.0546842}\\
4 & 0.0002097 & 0.6390678 & 100.03122 & 99.91623 & 199.9894 & 199.9474 & 0.0419379 & 0.0419379\\
\cellcolor{gray!6}{5} & \cellcolor{gray!6}{-0.0000656} & \cellcolor{gray!6}{0.8833763} & \cellcolor{gray!6}{100.06148} & \cellcolor{gray!6}{99.98887} & \cellcolor{gray!6}{200.0372} & \cellcolor{gray!6}{200.0504} & \cellcolor{gray!6}{-0.0131237} & \cellcolor{gray!6}{-0.0131237}\\
\addlinespace
6 & -0.0000562 & 0.8999136 & 100.04036 & 99.87487 & 199.9040 & 199.9152 & -0.0112445 & -0.0112445\\
\cellcolor{gray!6}{7} & \cellcolor{gray!6}{0.0000587} & \cellcolor{gray!6}{0.8955571} & \cellcolor{gray!6}{99.88474} & \cellcolor{gray!6}{99.98465} & \cellcolor{gray!6}{199.8811} & \cellcolor{gray!6}{199.8694} & \cellcolor{gray!6}{0.0117340} & \cellcolor{gray!6}{0.0117340}\\
9 & 0.0000324 & 0.9422655 & 100.18557 & 100.01365 & 200.2057 & 200.1992 & 0.0064841 & 0.0064841\\
\cellcolor{gray!6}{10} & \cellcolor{gray!6}{0.0004657} & \cellcolor{gray!6}{0.2977174} & \cellcolor{gray!6}{100.00307} & \cellcolor{gray!6}{99.91771} & \cellcolor{gray!6}{200.0139} & \cellcolor{gray!6}{199.9208} & \cellcolor{gray!6}{0.0931035} & \cellcolor{gray!6}{0.0931035}\\
11 & -0.0004638 & 0.2997043 & 99.94262 & 99.95997 & 199.8099 & 199.9026 & -0.0927130 & -0.0927130\\
\addlinespace
\cellcolor{gray!6}{12} & \cellcolor{gray!6}{-0.0002869} & \cellcolor{gray!6}{0.5211490} & \cellcolor{gray!6}{100.06612} & \cellcolor{gray!6}{100.07150} & \cellcolor{gray!6}{200.0802} & \cellcolor{gray!6}{200.1376} & \cellcolor{gray!6}{-0.0574237} & \cellcolor{gray!6}{-0.0574237}\\
13 & 0.0003968 & 0.3749637 & 99.93863 & 100.00096 & 200.0189 & 199.9396 & 0.0793309 & 0.0793309\\
\cellcolor{gray!6}{14} & \cellcolor{gray!6}{0.0008604} & \cellcolor{gray!6}{0.0543707} & \cellcolor{gray!6}{99.98128} & \cellcolor{gray!6}{99.98778} & \cellcolor{gray!6}{200.1411} & \cellcolor{gray!6}{199.9691} & \cellcolor{gray!6}{0.1720497} & \cellcolor{gray!6}{0.1720497}\\
15 & 0.0004709 & 0.2923753 & 99.98044 & 100.02723 & 200.1019 & 200.0077 & 0.0941801 & 0.0941801\\
\cellcolor{gray!6}{16} & \cellcolor{gray!6}{-0.0000638} & \cellcolor{gray!6}{0.8866115} & \cellcolor{gray!6}{99.98576} & \cellcolor{gray!6}{99.90296} & \cellcolor{gray!6}{199.8760} & \cellcolor{gray!6}{199.8887} & \cellcolor{gray!6}{-0.0127468} & \cellcolor{gray!6}{-0.0127468}\\
\addlinespace
17 & -0.0001013 & 0.8207471 & 100.08097 & 100.07135 & 200.1320 & 200.1523 & -0.0202817 & -0.0202817\\
\cellcolor{gray!6}{18} & \cellcolor{gray!6}{-0.0003408} & \cellcolor{gray!6}{0.4460659} & \cellcolor{gray!6}{100.08435} & \cellcolor{gray!6}{99.94296} & \cellcolor{gray!6}{199.9591} & \cellcolor{gray!6}{200.0273} & \cellcolor{gray!6}{-0.0681638} & \cellcolor{gray!6}{-0.0681638}\\
19 & -0.0005406 & 0.2267052 & 99.97086 & 99.75277 & 199.6156 & 199.7236 & -0.1079769 & -0.1079769\\
\cellcolor{gray!6}{20} & \cellcolor{gray!6}{-0.0002751} & \cellcolor{gray!6}{0.5385090} & \cellcolor{gray!6}{100.00111} & \cellcolor{gray!6}{99.96321} & \cellcolor{gray!6}{199.9093} & \cellcolor{gray!6}{199.9643} & \cellcolor{gray!6}{-0.0550037} & \cellcolor{gray!6}{-0.0550037}\\
21 & -0.0000966 & 0.8289283 & 99.79315 & 100.03184 & 199.8057 & 199.8250 & -0.0193095 & -0.0193095\\
\addlinespace
\cellcolor{gray!6}{22} & \cellcolor{gray!6}{-0.0004900} & \cellcolor{gray!6}{0.2732492} & \cellcolor{gray!6}{100.04193} & \cellcolor{gray!6}{99.81521} & \cellcolor{gray!6}{199.7592} & \cellcolor{gray!6}{199.8571} & \cellcolor{gray!6}{-0.0979244} & \cellcolor{gray!6}{-0.0979244}\\
23 & -0.0000384 & 0.9315195 & 99.98269 & 99.99324 & 199.9682 & 199.9759 & -0.0076852 & -0.0076852\\
\cellcolor{gray!6}{24} & \cellcolor{gray!6}{0.0000409} & \cellcolor{gray!6}{0.9270843} & \cellcolor{gray!6}{100.09771} & \cellcolor{gray!6}{100.04714} & \cellcolor{gray!6}{200.1530} & \cellcolor{gray!6}{200.1448} & \cellcolor{gray!6}{0.0081912} & \cellcolor{gray!6}{0.0081912}\\
25 & 0.0001548 & 0.7292396 & 100.02626 & 99.98853 & 200.0458 & 200.0148 & 0.0309617 & 0.0309617\\
\cellcolor{gray!6}{26} & \cellcolor{gray!6}{0.0000006} & \cellcolor{gray!6}{0.9989683} & \cellcolor{gray!6}{99.91310} & \cellcolor{gray!6}{99.97783} & \cellcolor{gray!6}{199.8911} & \cellcolor{gray!6}{199.8909} & \cellcolor{gray!6}{0.0001156} & \cellcolor{gray!6}{0.0001156}\\
\addlinespace
27 & -0.0004062 & 0.3636902 & 99.95707 & 99.97077 & 199.8466 & 199.9278 & -0.0812165 & -0.0812165\\
\cellcolor{gray!6}{28} & \cellcolor{gray!6}{0.0004554} & \cellcolor{gray!6}{0.3085175} & \cellcolor{gray!6}{99.97648} & \cellcolor{gray!6}{100.03075} & \cellcolor{gray!6}{200.0983} & \cellcolor{gray!6}{200.0072} & \cellcolor{gray!6}{0.0910862} & \cellcolor{gray!6}{0.0910862}\\
29 & -0.0002536 & 0.5707257 & 99.97185 & 99.96773 & 199.8889 & 199.9396 & -0.0506972 & -0.0506972\\
\cellcolor{gray!6}{30} & \cellcolor{gray!6}{0.0005116} & \cellcolor{gray!6}{0.2525925} & \cellcolor{gray!6}{100.07632} & \cellcolor{gray!6}{100.03861} & \cellcolor{gray!6}{200.2173} & \cellcolor{gray!6}{200.1149} & \cellcolor{gray!6}{0.1023880} & \cellcolor{gray!6}{0.1023880}\\
31 & -0.0000701 & 0.8755172 & 100.09967 & 99.96810 & 200.0537 & 200.0678 & -0.0140163 & -0.0140163\\
\addlinespace
\cellcolor{gray!6}{32} & \cellcolor{gray!6}{0.0002323} & \cellcolor{gray!6}{0.6033995} & \cellcolor{gray!6}{99.82793} & \cellcolor{gray!6}{100.03516} & \cellcolor{gray!6}{199.9095} & \cellcolor{gray!6}{199.8631} & \cellcolor{gray!6}{0.0464353} & \cellcolor{gray!6}{0.0464353}\\
33 & 0.0007507 & 0.0932231 & 100.09395 & 99.96132 & 200.2055 & 200.0553 & 0.1501832 & 0.1501832\\
\cellcolor{gray!6}{34} & \cellcolor{gray!6}{-0.0002231} & \cellcolor{gray!6}{0.6178539} & \cellcolor{gray!6}{99.77223} & \cellcolor{gray!6}{100.08013} & \cellcolor{gray!6}{199.8078} & \cellcolor{gray!6}{199.8524} & \cellcolor{gray!6}{-0.0445895} & \cellcolor{gray!6}{-0.0445895}\\
35 & 0.0003078 & 0.4913172 & 100.05562 & 99.93641 & 200.0536 & 199.9920 & 0.0615534 & 0.0615534\\
\cellcolor{gray!6}{36} & \cellcolor{gray!6}{0.0000576} & \cellcolor{gray!6}{0.8974954} & \cellcolor{gray!6}{99.94412} & \cellcolor{gray!6}{99.96231} & \cellcolor{gray!6}{199.9180} & \cellcolor{gray!6}{199.9064} & \cellcolor{gray!6}{0.0115172} & \cellcolor{gray!6}{0.0115172}\\
\addlinespace
37 & -0.0006663 & 0.1362344 & 99.98475 & 99.93338 & 199.7849 & 199.9181 & -0.1332117 & -0.1332117\\
\cellcolor{gray!6}{38} & \cellcolor{gray!6}{-0.0000595} & \cellcolor{gray!6}{0.8942027} & \cellcolor{gray!6}{100.13083} & \cellcolor{gray!6}{100.05302} & \cellcolor{gray!6}{200.1719} & \cellcolor{gray!6}{200.1838} & \cellcolor{gray!6}{-0.0119058} & \cellcolor{gray!6}{-0.0119058}\\
39 & 0.0005318 & 0.2343975 & 100.00880 & 99.95418 & 200.0693 & 199.9630 & 0.1063373 & 0.1063373\\
\cellcolor{gray!6}{40} & \cellcolor{gray!6}{-0.0001554} & \cellcolor{gray!6}{0.7281648} & \cellcolor{gray!6}{100.10111} & \cellcolor{gray!6}{99.86696} & \cellcolor{gray!6}{199.9370} & \cellcolor{gray!6}{199.9681} & \cellcolor{gray!6}{-0.0310824} & \cellcolor{gray!6}{-0.0310824}\\
41 & 0.0002603 & 0.5604638 & 99.93540 & 100.00392 & 199.9914 & 199.9393 & 0.0520535 & 0.0520535\\
\addlinespace
\cellcolor{gray!6}{42} & \cellcolor{gray!6}{0.0002006} & \cellcolor{gray!6}{0.6537639} & \cellcolor{gray!6}{100.09478} & \cellcolor{gray!6}{99.95661} & \cellcolor{gray!6}{200.0915} & \cellcolor{gray!6}{200.0514} & \cellcolor{gray!6}{0.0401289} & \cellcolor{gray!6}{0.0401289}\\
43 & 0.0005878 & 0.1887052 & 100.10992 & 99.98966 & 200.2172 & 200.0996 & 0.1176238 & 0.1176238\\
\cellcolor{gray!6}{44} & \cellcolor{gray!6}{0.0000720} & \cellcolor{gray!6}{0.8720832} & \cellcolor{gray!6}{100.00396} & \cellcolor{gray!6}{99.89731} & \cellcolor{gray!6}{199.9157} & \cellcolor{gray!6}{199.9013} & \cellcolor{gray!6}{0.0143943} & \cellcolor{gray!6}{0.0143943}\\
45 & -0.0002899 & 0.5168766 & 100.04341 & 99.96053 & 199.9460 & 200.0039 & -0.0579750 & -0.0579750\\
\bottomrule
\end{tabular}}
\end{table}

\begin{table}

\caption{\label{tab:unnamed-chunk-5}ランダムサンプリングしたデータが独立でない場合}
\centering
\resizebox{\linewidth}{!}{
\begin{tabular}[t]{rrrrrrrrr}
\toprule
No & 相関係数 & p値 & 標本x & 標本y & 加法1 & 加法2 & 差異 & cov2\\
\midrule
\cellcolor{gray!6}{8} & \cellcolor{gray!6}{0.0011228} & \cellcolor{gray!6}{0.0120532} & \cellcolor{gray!6}{100.0812} & \cellcolor{gray!6}{100.2103} & \cellcolor{gray!6}{200.5164} & \cellcolor{gray!6}{200.2915} & \cellcolor{gray!6}{0.2248807} & \cellcolor{gray!6}{0.2248807}\\
\bottomrule
\end{tabular}}
\end{table}

\[\mbox{加法1} = var(x + y), \mbox{加法2} = var(x) + var(y)\]

\newpage

\hypertarget{ux307eux3068ux3081}{%
\subsection{まとめ}\label{ux307eux3068ux3081}}

 データが独立であれば分散の加法性が成り立っていることがわかります。データが独立とは言い難い無相関の検定が成功するケース(\(95\%\)信頼区間に\(0\)が入らない)では、分散の差(共分散の2倍数)が一桁大きいので加法性が成り立っているとは言い難いように思えますがこのケースでは数値だけを見ている限り差はよくわかりません。

 

\hypertarget{cor.testux95a2ux6570ux306bux3064ux3044ux3066}{%
\subsection{\texorpdfstring{\texttt{cor.test()}関数について}{cor.test()関数について}}\label{cor.testux95a2ux6570ux306bux3064ux3044ux3066}}

 \texttt{cor.test()}関数は無相関の検定を行う関数です。対立仮説(\(H_1\))は下記の出力の通り「true
correlation is \textbf{not} equal to
0(相関係数はゼロではない)」ですので、帰無仮説(\(H_0\))は「相関係数はゼロである(相関はない)」となります。有意水準\(\alpha\)で検定が失敗すれば(帰無仮説が棄却されない、\(p \geqq \alpha\)である)帰無仮説が採択されますので相関係数はゼロ(データ間には相関がない)と考えられます。

\begin{verbatim}
## 
##  Pearson's product-moment correlation
## 
## data:  rnorm(n) and rnorm(n)
## t = -0.71677, df = 4999998, p-value = 0.4735
## alternative hypothesis: true correlation is not equal to 0
## 95 percent confidence interval:
##  -0.0011970704  0.0005559746
## sample estimates:
##           cor 
## -0.0003205482
\end{verbatim}

 

\hypertarget{appendix}{%
\section{Appendix}\label{appendix}}

\hypertarget{about-handout-style}{%
\subsection{About handout style}\label{about-handout-style}}

The Tufte handout style is a style that Edward Tufte uses in his books
and handouts. Tufte's style is known for its extensive use of sidenotes,
tight integration of graphics with text, and well-set typography. This
style has been implemented in LaTeX and HTML/CSS\footnote{See Github
  repositories
  \href{https://github.com/tufte-latex/tufte-latex}{tufte-latex} and
  \href{https://github.com/edwardtufte/tufte-css}{tufte-css}},
respectively.

 

\bibliography{bib/references.bib}



\end{document}

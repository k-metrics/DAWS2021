\documentclass[]{tufte-handout}

% ams
\usepackage{amssymb,amsmath}

\usepackage{ifxetex,ifluatex}
\usepackage{fixltx2e} % provides \textsubscript
\ifnum 0\ifxetex 1\fi\ifluatex 1\fi=0 % if pdftex
  \usepackage[T1]{fontenc}
  \usepackage[utf8]{inputenc}
\else % if luatex or xelatex
  \makeatletter
  \@ifpackageloaded{fontspec}{}{\usepackage{fontspec}}
  \makeatother
  \defaultfontfeatures{Ligatures=TeX,Scale=MatchLowercase}
  \makeatletter
  \@ifpackageloaded{soul}{
     \renewcommand\allcapsspacing[1]{{\addfontfeature{LetterSpace=15}#1}}
     \renewcommand\smallcapsspacing[1]{{\addfontfeature{LetterSpace=10}#1}}
   }{}
  \makeatother

\fi

% graphix
\usepackage{graphicx}
\setkeys{Gin}{width=\linewidth,totalheight=\textheight,keepaspectratio}

% booktabs
\usepackage{booktabs}

% url
\usepackage{url}

% hyperref
\usepackage{hyperref}

% units.
\usepackage{units}


\setcounter{secnumdepth}{-1}

% citations
\usepackage{natbib}
\bibliographystyle{plainnat}


% pandoc syntax highlighting
\usepackage{color}
\usepackage{fancyvrb}
\newcommand{\VerbBar}{|}
\newcommand{\VERB}{\Verb[commandchars=\\\{\}]}
\DefineVerbatimEnvironment{Highlighting}{Verbatim}{commandchars=\\\{\}}
% Add ',fontsize=\small' for more characters per line
\newenvironment{Shaded}{}{}
\newcommand{\AlertTok}[1]{\textcolor[rgb]{1.00,0.00,0.00}{\textbf{#1}}}
\newcommand{\AnnotationTok}[1]{\textcolor[rgb]{0.38,0.63,0.69}{\textbf{\textit{#1}}}}
\newcommand{\AttributeTok}[1]{\textcolor[rgb]{0.49,0.56,0.16}{#1}}
\newcommand{\BaseNTok}[1]{\textcolor[rgb]{0.25,0.63,0.44}{#1}}
\newcommand{\BuiltInTok}[1]{#1}
\newcommand{\CharTok}[1]{\textcolor[rgb]{0.25,0.44,0.63}{#1}}
\newcommand{\CommentTok}[1]{\textcolor[rgb]{0.38,0.63,0.69}{\textit{#1}}}
\newcommand{\CommentVarTok}[1]{\textcolor[rgb]{0.38,0.63,0.69}{\textbf{\textit{#1}}}}
\newcommand{\ConstantTok}[1]{\textcolor[rgb]{0.53,0.00,0.00}{#1}}
\newcommand{\ControlFlowTok}[1]{\textcolor[rgb]{0.00,0.44,0.13}{\textbf{#1}}}
\newcommand{\DataTypeTok}[1]{\textcolor[rgb]{0.56,0.13,0.00}{#1}}
\newcommand{\DecValTok}[1]{\textcolor[rgb]{0.25,0.63,0.44}{#1}}
\newcommand{\DocumentationTok}[1]{\textcolor[rgb]{0.73,0.13,0.13}{\textit{#1}}}
\newcommand{\ErrorTok}[1]{\textcolor[rgb]{1.00,0.00,0.00}{\textbf{#1}}}
\newcommand{\ExtensionTok}[1]{#1}
\newcommand{\FloatTok}[1]{\textcolor[rgb]{0.25,0.63,0.44}{#1}}
\newcommand{\FunctionTok}[1]{\textcolor[rgb]{0.02,0.16,0.49}{#1}}
\newcommand{\ImportTok}[1]{#1}
\newcommand{\InformationTok}[1]{\textcolor[rgb]{0.38,0.63,0.69}{\textbf{\textit{#1}}}}
\newcommand{\KeywordTok}[1]{\textcolor[rgb]{0.00,0.44,0.13}{\textbf{#1}}}
\newcommand{\NormalTok}[1]{#1}
\newcommand{\OperatorTok}[1]{\textcolor[rgb]{0.40,0.40,0.40}{#1}}
\newcommand{\OtherTok}[1]{\textcolor[rgb]{0.00,0.44,0.13}{#1}}
\newcommand{\PreprocessorTok}[1]{\textcolor[rgb]{0.74,0.48,0.00}{#1}}
\newcommand{\RegionMarkerTok}[1]{#1}
\newcommand{\SpecialCharTok}[1]{\textcolor[rgb]{0.25,0.44,0.63}{#1}}
\newcommand{\SpecialStringTok}[1]{\textcolor[rgb]{0.73,0.40,0.53}{#1}}
\newcommand{\StringTok}[1]{\textcolor[rgb]{0.25,0.44,0.63}{#1}}
\newcommand{\VariableTok}[1]{\textcolor[rgb]{0.10,0.09,0.49}{#1}}
\newcommand{\VerbatimStringTok}[1]{\textcolor[rgb]{0.25,0.44,0.63}{#1}}
\newcommand{\WarningTok}[1]{\textcolor[rgb]{0.38,0.63,0.69}{\textbf{\textit{#1}}}}

% longtable
\usepackage{longtable,booktabs}

% multiplecol
\usepackage{multicol}

% strikeout
\usepackage[normalem]{ulem}

% morefloats
\usepackage{morefloats}


% tightlist macro required by pandoc >= 1.14
\providecommand{\tightlist}{%
  \setlength{\itemsep}{0pt}\setlength{\parskip}{0pt}}

% title / author / date
\title[分散の加法性を視覚的に理解する(その3)]{分散の加法性を視覚的に理解する(その3)}
\author{Sampo Suzuki, CC 4.0 BY-NC-SA}
\date{2021-06-01}

% --- 参考資料 ----------------------------------------------------------------
% https://github.com/Gedevan-Aleksizde/Japan.R2019/blob/master/latex/preamble.tex
% https://teastat.blogspot.com/2019/01/bookdown.html

% --- Packages ----------------------------------------------------------------
% 日本語とtufte, kableExtraを使うために必要なTeXパッケージ指定
% tufteではA4サイズの指定が不可能
%  A4 210mm x 297mm
%   \usepackage[a4paper, total={6.5in, 9.5in}]{geometry}
%   \usepackage{indentfirst}   # tinytexのリポジトリには存在しない?
% \usepackage[a4paper, total={160mm, 247mm}, left=25mm, top=25mm]{geometry}
% \usepackage[pdfbox,tombo]{gentombow}  % トンボを設定する場合は有効にする
\usepackage{ifthen}                     % 条件分岐用 \ifthenelse{条件}{T}{F}
\usepackage{booktabs}                   % ここからkableExtra用パッケージ
\usepackage{longtable}                  % 
\usepackage{array}                      % 
\usepackage{multirow}                   % 
\usepackage{wrapfig}                    % 
\usepackage{float}                      % 
\usepackage{colortbl}                   % 
\usepackage{pdflscape}                  % 
\usepackage{tabu}                       % 
\usepackage{threeparttable}             % 
\usepackage{threeparttablex}            % 
\usepackage[normalem]{ulem}             % 
\usepackage{inputenc}                   % 
\usepackage{makecell}                   % 
\usepackage{xcolor}                     % ここまでkableExtra用
\usepackage{amsmath}                    % 
\usepackage{fontawesome5}               % fontawesomeを使うために必要
\usepackage{subfig}                     % 複数の図を並べる際に必要(古い?)
% \usepackage{subcaption}                 % 同上(新しい?)
\usepackage{xeCJK}                      % 以下、日本語フォント用に必要
\usepackage[noto]{zxjafont}             % Linux環境ではこちを指定
% \usepackage[haranoaji]{zxjafont}      % Windows環境ではこちらを指定する
\usepackage{zxjatype}                   % 日本語処理に必要
\usepackage{pxrubrica}                  % ルビ用
\usepackage{hyperref}                   % ハイパーリンク用必要?

% --- Index ------------------------------------------------------------------
% https://texwiki.texjp.org/?%E7%B4%A2%E5%BC%95%E4%BD%9C%E6%88%90
% これを指定するとIndex(索引)は作成されるが参照ページがズレる
% 中間ファイルの.indではページはズレていないので、その後の結合処理がおかしい
% \usepackage{makeidx}
% \makeindex
% \usepackage{showidx}                  % 索引確認用

% --- Table of Contentes ------------------------------------------------------
% TOCにLOT(List of Tables), LOF(List of Figures), Bibliography, Indexを表示
% \usepackage[nottoc]{tocbibind}

% --- Fonts -------------------------------------------------------------------
% フォントしては index.html でも可能(pandoc用オプションは index.htmlにて)
% \setCJKmonofont{Source Han Code JP}
\setmonofont{Source Han Code JP}     % Linuxではこれのみコメントアウトする
% \setjamonofont{Source Han Code JP}

% ## 日本語フォントの扱いについてはzxjafontパッケージの解説を参照のこと
% # https://mirror.las.iastate.edu/tex-archive/language/japanese/zxjafont/zxjafont.pdf
% #
% ## Windows環境ではなぜかNotoフォントが認識されないので源ノシリーズベースの
% ## 原ノ味フォントかIPAexフォントを利用する(原ノ味はtlmgrでインストール可)
% # \usepackage[haranoaji]{zxjafont}
% # \usepackage[ipaex]{zxjafont}
% #
% ## Windows環境でNotoフォントを指定したい場合は以下のようにheader-includeで
% ## 個別に指定する(setCJKxxxfotnの指定は必要?)
% # \setmainfont{NotoSerifCJKjp-Regular.otf}[BoldFont=NotoSerifCJKjp-Bold.otf]
% # \setsansfont{NotoSansCJKjp-Regular.otf}[BoldFont=NotoSansCJKjp-Bold.otf]
% # \setmonofont{NotoSansMonoCJKjp-Regular.otf}[BoldFont=NotoSansMonoCJKjp-Bold.otf]
% ## モノフォントは源ノ角コード(Source Code Proの日本語版)がおすゝめ
% # \setmonofont{SourceHanCodeJP-Regular.otf}[BoldFont=SourceHanCodeJPS-Bold.otf]

\begin{document}

\maketitle




\hypertarget{ux306fux3058ux3081ux306b}{%
\section{\texorpdfstring{\textbf{はじめに}}{はじめに}}\label{ux306fux3058ux3081ux306b}}

 分散の加法性を視覚的に理解する(その2)において、データが独立であれば分散の加法性がなりたつことがわかりました。では、同一正規分布から取り出した二つ、および、三つの値の平均値の場合はどうなるか、その2と同様の手段で確認してみます。

 

\hypertarget{ux540cux4e00ux30c7ux30fcux30bfux304bux3089ux30b5ux30f3ux30d7ux30eaux30f3ux30b0ux3057ux305fux4e8cux3064ux306eux5024ux3092ux5e73ux5747ux3057ux305fux5834ux5408}{%
\subsection{\texorpdfstring{\textbf{同一データからサンプリングした二つの値を平均した場合}}{同一データからサンプリングした二つの値を平均した場合}}\label{ux540cux4e00ux30c7ux30fcux30bfux304bux3089ux30b5ux30f3ux30d7ux30eaux30f3ux30b0ux3057ux305fux4e8cux3064ux306eux5024ux3092ux5e73ux5747ux3057ux305fux5834ux5408}}

 最初に以下の処理を行う関数を定義します。

\begin{itemize}
\tightlist
\item
  データを乱数生成する\footnote{今回は\texttt{rnorm()}関数による分散が\(100\)となる正規分布}
\item
  乱数生成したデータをランダムサンプリングする
\item
  作成したデータの統計量を求める
\item
  無相関検定の結果と統計量をデータフレームにまとめる
\end{itemize}

\begin{Shaded}
\begin{Highlighting}[numbers=left,,]
\NormalTok{f2 }\OtherTok{\textless{}{-}} \ControlFlowTok{function}\NormalTok{(}\AttributeTok{i =} \ConstantTok{NA}\NormalTok{, }\AttributeTok{n =} \DecValTok{5000000}\NormalTok{) \{}
  \CommentTok{\# データを乱数生成する}
\NormalTok{  x }\OtherTok{\textless{}{-}} \FunctionTok{rnorm}\NormalTok{(}\AttributeTok{n =}\NormalTok{ n, }\AttributeTok{mean =} \DecValTok{10}\NormalTok{, }\AttributeTok{sd =} \DecValTok{10}\NormalTok{)}
  \CommentTok{\# 乱数生成したデータから二つのデータを取り出す}
\NormalTok{  a }\OtherTok{\textless{}{-}} \FunctionTok{sample}\NormalTok{(x, n, }\AttributeTok{replace =} \ConstantTok{TRUE}\NormalTok{)}
\NormalTok{  b }\OtherTok{\textless{}{-}} \FunctionTok{sample}\NormalTok{(x, n, }\AttributeTok{replace =} \ConstantTok{TRUE}\NormalTok{)}
\NormalTok{  num }\OtherTok{\textless{}{-}} \DecValTok{2}
  \CommentTok{\# 統計量を求める}
\NormalTok{  df }\OtherTok{\textless{}{-}} \FunctionTok{data.frame}\NormalTok{(}\AttributeTok{no =}\NormalTok{ i,}
                   \AttributeTok{var.x =} \FunctionTok{var}\NormalTok{(x),}
                   \AttributeTok{var.a =} \FunctionTok{var}\NormalTok{(a), }\AttributeTok{var.b =} \FunctionTok{var}\NormalTok{(b),}
                   \AttributeTok{var.ab =} \FunctionTok{var}\NormalTok{((a }\SpecialCharTok{+}\NormalTok{ b) }\SpecialCharTok{/}\NormalTok{ num), }\AttributeTok{var.sum =}\NormalTok{ (}\FunctionTok{var}\NormalTok{(a }\SpecialCharTok{/}\NormalTok{ num) }\SpecialCharTok{+} \FunctionTok{var}\NormalTok{(b }\SpecialCharTok{/}\NormalTok{ num)),}
                   \AttributeTok{cov =} \FunctionTok{cov}\NormalTok{(a }\SpecialCharTok{/}\NormalTok{ num, b }\SpecialCharTok{/}\NormalTok{ num ),}
                   \AttributeTok{cov2 =} \FunctionTok{cov}\NormalTok{(a }\SpecialCharTok{/}\NormalTok{ num, b }\SpecialCharTok{/}\NormalTok{ num) }\SpecialCharTok{*} \DecValTok{2}\NormalTok{)}
  \CommentTok{\# 無相関の検定結果と統計量をデータフレームにまとめる}
\NormalTok{  df }\OtherTok{\textless{}{-}} \FunctionTok{cor.test}\NormalTok{(a, b) }\SpecialCharTok{\%\textgreater{}\%}\NormalTok{ broom}\SpecialCharTok{::}\FunctionTok{tidy}\NormalTok{() }\SpecialCharTok{\%\textgreater{}\%}\NormalTok{ dplyr}\SpecialCharTok{::}\FunctionTok{bind\_cols}\NormalTok{(df)}
  \FunctionTok{return}\NormalTok{(df)}
\NormalTok{\}}
\end{Highlighting}
\end{Shaded}

\begin{longtable}[]{@{}rrrrrrrrrrr@{}}
\caption{二つのサンプルを平均した場合の分散}\tabularnewline
\toprule
No & 相関係数 & p値 & 母集団 & 標本a & 標本b & 加法1 & 加法2 & 差異 &
母集団比 & cov2 \\
\midrule
\endfirsthead
\toprule
No & 相関係数 & p値 & 母集団 & 標本a & 標本b & 加法1 & 加法2 & 差異 &
母集団比 & cov2 \\
\midrule
\endhead
2 & 0.000 & 0.358 & 99.870 & 99.909 & 99.957 & 49.987 & 49.967 & 0.021 &
0.501 & 0.021 \\
4 & 0.000 & 0.842 & 100.079 & 100.129 & 100.043 & 50.038 & 50.043 &
-0.004 & 0.500 & -0.004 \\
5 & -0.001 & 0.178 & 100.086 & 100.129 & 100.016 & 50.006 & 50.036 &
-0.030 & 0.500 & -0.030 \\
6 & 0.000 & 0.322 & 100.059 & 100.070 & 100.064 & 50.011 & 50.034 &
-0.022 & 0.500 & -0.022 \\
7 & 0.000 & 0.551 & 99.974 & 99.993 & 99.896 & 49.959 & 49.972 & -0.013
& 0.500 & -0.013 \\
8 & 0.000 & 0.684 & 100.021 & 99.948 & 100.013 & 49.999 & 49.990 & 0.009
& 0.500 & 0.009 \\
9 & 0.000 & 0.604 & 100.022 & 100.032 & 99.981 & 49.991 & 50.003 &
-0.012 & 0.500 & -0.012 \\
10 & 0.001 & 0.197 & 99.869 & 99.844 & 99.867 & 49.957 & 49.928 & 0.029
& 0.500 & 0.029 \\
11 & 0.000 & 0.628 & 100.059 & 100.149 & 99.999 & 50.048 & 50.037 &
0.011 & 0.500 & 0.011 \\
12 & 0.000 & 0.570 & 99.988 & 99.986 & 99.940 & 49.994 & 49.982 & 0.013
& 0.500 & 0.013 \\
13 & 0.000 & 0.783 & 99.986 & 100.023 & 99.891 & 49.972 & 49.979 &
-0.006 & 0.500 & -0.006 \\
16 & 0.000 & 0.555 & 100.025 & 100.083 & 99.997 & 50.007 & 50.020 &
-0.013 & 0.500 & -0.013 \\
17 & 0.000 & 0.718 & 100.092 & 100.200 & 100.107 & 50.069 & 50.077 &
-0.008 & 0.500 & -0.008 \\
18 & 0.000 & 0.493 & 99.936 & 99.861 & 100.044 & 49.992 & 49.976 & 0.015
& 0.500 & 0.015 \\
19 & 0.000 & 0.613 & 100.071 & 99.955 & 100.036 & 49.986 & 49.998 &
-0.011 & 0.500 & -0.011 \\
20 & 0.000 & 0.383 & 99.987 & 99.966 & 99.947 & 49.998 & 49.978 & 0.020
& 0.500 & 0.020 \\
21 & 0.000 & 0.401 & 100.018 & 100.084 & 100.004 & 50.003 & 50.022 &
-0.019 & 0.500 & -0.019 \\
22 & 0.000 & 0.884 & 100.019 & 99.932 & 100.023 & 49.992 & 49.989 &
0.003 & 0.500 & 0.003 \\
23 & 0.000 & 0.709 & 100.010 & 100.011 & 100.098 & 50.036 & 50.027 &
0.008 & 0.500 & 0.008 \\
24 & 0.000 & 0.452 & 100.051 & 99.943 & 100.129 & 50.035 & 50.018 &
0.017 & 0.500 & 0.017 \\
25 & 0.000 & 0.673 & 100.014 & 100.075 & 100.003 & 50.010 & 50.019 &
-0.009 & 0.500 & -0.009 \\
26 & 0.000 & 0.416 & 100.010 & 99.966 & 99.993 & 49.972 & 49.990 &
-0.018 & 0.500 & -0.018 \\
27 & 0.000 & 0.744 & 100.028 & 100.095 & 100.036 & 50.040 & 50.033 &
0.007 & 0.500 & 0.007 \\
28 & -0.001 & 0.180 & 99.916 & 99.992 & 99.932 & 49.951 & 49.981 &
-0.030 & 0.500 & -0.030 \\
29 & 0.001 & 0.169 & 99.979 & 100.079 & 99.971 & 50.043 & 50.012 & 0.031
& 0.501 & 0.031 \\
30 & 0.001 & 0.185 & 100.156 & 100.189 & 100.185 & 50.123 & 50.094 &
0.030 & 0.500 & 0.030 \\
\bottomrule
\end{longtable}

\begin{longtable}[]{@{}rrrrrrrrrrr@{}}
\caption{二つのサンプルが独立でない場合}\tabularnewline
\toprule
No & 相関係数 & p値 & 母集団 & 標本a & 標本b & 加法1 & 加法2 & 差異 &
母集団比 & cov2 \\
\midrule
\endfirsthead
\toprule
No & 相関係数 & p値 & 母集団 & 標本a & 標本b & 加法1 & 加法2 & 差異 &
母集団比 & cov2 \\
\midrule
\endhead
1 & -0.001 & 0.045 & 100.011 & 100.128 & 100.048 & 49.999 & 50.044 &
-0.045 & 0.500 & -0.045 \\
3 & 0.001 & 0.033 & 100.040 & 100.097 & 100.066 & 50.088 & 50.041 &
0.048 & 0.501 & 0.048 \\
14 & -0.001 & 0.013 & 100.070 & 100.096 & 100.076 & 49.987 & 50.043 &
-0.056 & 0.500 & -0.056 \\
15 & 0.001 & 0.049 & 100.106 & 100.067 & 100.094 & 50.084 & 50.040 &
0.044 & 0.500 & 0.044 \\
\bottomrule
\end{longtable}

\[\mbox{加法1} = var(\frac{a + b}{2}), \mbox{加法2} = var(\frac{a}{2}) + var(\frac{b}{2})\]

\newpage

\hypertarget{ux540cux4e00ux30c7ux30fcux30bfux304bux3089ux30b5ux30f3ux30d7ux30eaux30f3ux30b0ux3057ux305fux4e09ux3064ux306eux5024ux3092ux5e73ux5747ux3057ux305fux5834ux5408}{%
\subsection{\texorpdfstring{\textbf{同一データからサンプリングした三つの値を平均した場合}}{同一データからサンプリングした三つの値を平均した場合}}\label{ux540cux4e00ux30c7ux30fcux30bfux304bux3089ux30b5ux30f3ux30d7ux30eaux30f3ux30b0ux3057ux305fux4e09ux3064ux306eux5024ux3092ux5e73ux5747ux3057ux305fux5834ux5408}}

 最初に以下の処理を行う関数を定義します。

\begin{itemize}
\tightlist
\item
  データを乱数生成する\footnote{今回は\texttt{rnorm()}関数による分散が\(100\)となる正規分布}
\item
  乱数生成したデータをランダムサンプリングする
\item
  作成したデータの統計量を求める
\item
  無相関検定の結果と統計量をデータフレームにまとめる
\end{itemize}

\begin{Shaded}
\begin{Highlighting}[numbers=left,,]
\NormalTok{f3 }\OtherTok{\textless{}{-}} \ControlFlowTok{function}\NormalTok{(}\AttributeTok{i =} \ConstantTok{NA}\NormalTok{, }\AttributeTok{n =} \DecValTok{5000000}\NormalTok{) \{}
  \CommentTok{\# データを乱数生成する}
\NormalTok{  x }\OtherTok{\textless{}{-}} \FunctionTok{rnorm}\NormalTok{(}\AttributeTok{n =}\NormalTok{ n, }\AttributeTok{mean =} \DecValTok{10}\NormalTok{, }\AttributeTok{sd =} \DecValTok{10}\NormalTok{)}
  \CommentTok{\# 乱数生成したデータから三つのデータを取り出す}
\NormalTok{  a }\OtherTok{\textless{}{-}} \FunctionTok{sample}\NormalTok{(x, n, }\AttributeTok{replace =} \ConstantTok{TRUE}\NormalTok{)}
\NormalTok{  b }\OtherTok{\textless{}{-}} \FunctionTok{sample}\NormalTok{(x, n, }\AttributeTok{replace =} \ConstantTok{TRUE}\NormalTok{)}
\NormalTok{  c }\OtherTok{\textless{}{-}} \FunctionTok{sample}\NormalTok{(x, n, }\AttributeTok{replace =} \ConstantTok{TRUE}\NormalTok{)}
\NormalTok{  num }\OtherTok{\textless{}{-}} \DecValTok{3}
  \CommentTok{\# 統計量を求める}
\NormalTok{  df }\OtherTok{\textless{}{-}} \FunctionTok{data.frame}\NormalTok{(}\AttributeTok{no =}\NormalTok{ i,}
                   \AttributeTok{var.x =} \FunctionTok{var}\NormalTok{(x), }
                   \AttributeTok{var.a =} \FunctionTok{var}\NormalTok{(a), }\AttributeTok{var.b =} \FunctionTok{var}\NormalTok{(b), }\AttributeTok{var.c =} \FunctionTok{var}\NormalTok{(c),}
                   \AttributeTok{var.abc =} \FunctionTok{var}\NormalTok{((a }\SpecialCharTok{+}\NormalTok{ b }\SpecialCharTok{+}\NormalTok{ c) }\SpecialCharTok{/}\NormalTok{ num),}
                   \AttributeTok{var.sum =}\NormalTok{ (}\FunctionTok{var}\NormalTok{(a }\SpecialCharTok{/}\NormalTok{ num) }\SpecialCharTok{+} \FunctionTok{var}\NormalTok{(b }\SpecialCharTok{/}\NormalTok{ num) }\SpecialCharTok{+} \FunctionTok{var}\NormalTok{(c }\SpecialCharTok{/}\NormalTok{ num)),}
                   \AttributeTok{cov.ab =} \FunctionTok{cov}\NormalTok{(a, b), }\AttributeTok{cov.ac =} \FunctionTok{cov}\NormalTok{(a, c), }\AttributeTok{cov.bc =} \FunctionTok{cov}\NormalTok{(b, c),}
                   \AttributeTok{cov2.ab =} \FunctionTok{cov}\NormalTok{(a, b) }\SpecialCharTok{*} \DecValTok{2}\NormalTok{, }\AttributeTok{cov2.ac =} \FunctionTok{cov}\NormalTok{(a, c) }\SpecialCharTok{*} \DecValTok{2}\NormalTok{, }\AttributeTok{cov2.bc =} \FunctionTok{cov}\NormalTok{(b, c) }\SpecialCharTok{*} \DecValTok{2}\NormalTok{)}
  \CommentTok{\# 無相関の検定結果と統計量をデータフレームにまとめる}
\NormalTok{  df }\OtherTok{\textless{}{-}} \FunctionTok{cor.test}\NormalTok{(a, b) }\SpecialCharTok{\%\textgreater{}\%}\NormalTok{ broom}\SpecialCharTok{::}\FunctionTok{tidy}\NormalTok{() }\SpecialCharTok{\%\textgreater{}\%}\NormalTok{ dplyr}\SpecialCharTok{::}\FunctionTok{bind\_cols}\NormalTok{(df)}
\NormalTok{  df }\OtherTok{\textless{}{-}} \FunctionTok{cor.test}\NormalTok{(a, c) }\SpecialCharTok{\%\textgreater{}\%}\NormalTok{ broom}\SpecialCharTok{::}\FunctionTok{tidy}\NormalTok{() }\SpecialCharTok{\%\textgreater{}\%}\NormalTok{ dplyr}\SpecialCharTok{::}\FunctionTok{bind\_cols}\NormalTok{(df)}
\NormalTok{  df }\OtherTok{\textless{}{-}} \FunctionTok{cor.test}\NormalTok{(b, c) }\SpecialCharTok{\%\textgreater{}\%}\NormalTok{ broom}\SpecialCharTok{::}\FunctionTok{tidy}\NormalTok{() }\SpecialCharTok{\%\textgreater{}\%}\NormalTok{ dplyr}\SpecialCharTok{::}\FunctionTok{bind\_cols}\NormalTok{(df)}
  \FunctionTok{return}\NormalTok{(df)}
\NormalTok{\}}
\end{Highlighting}
\end{Shaded}

\newpage

\begin{longtable}[]{@{}rrrrrrrrr@{}}
\caption{三つのサンプルを平均した場合の分散}\tabularnewline
\toprule
No & 母集団 & 標本a & 標本b & 標本c & 加法1 & 加法2 & 差異 & 母集団比 \\
\midrule
\endfirsthead
\toprule
No & 母集団 & 標本a & 標本b & 標本c & 加法1 & 加法2 & 差異 & 母集団比 \\
\midrule
\endhead
1 & 99.945 & 99.934 & 100.023 & 99.931 & 33.305 & 33.321 & -0.016 &
0.333 \\
3 & 99.973 & 99.938 & 99.870 & 99.907 & 33.306 & 33.302 & 0.005 &
0.333 \\
4 & 100.020 & 99.998 & 99.930 & 100.067 & 33.307 & 33.333 & -0.026 &
0.333 \\
5 & 99.947 & 99.905 & 100.021 & 99.909 & 33.319 & 33.315 & 0.004 &
0.333 \\
6 & 99.966 & 99.965 & 99.928 & 99.928 & 33.329 & 33.314 & 0.016 &
0.333 \\
7 & 100.002 & 100.063 & 100.001 & 99.956 & 33.351 & 33.335 & 0.016 &
0.334 \\
8 & 100.004 & 100.050 & 99.928 & 100.014 & 33.319 & 33.332 & -0.013 &
0.333 \\
9 & 100.079 & 100.021 & 100.057 & 100.095 & 33.365 & 33.353 & 0.013 &
0.333 \\
10 & 100.228 & 100.285 & 100.336 & 100.261 & 33.418 & 33.431 & -0.013 &
0.333 \\
11 & 100.006 & 100.048 & 100.064 & 100.030 & 33.355 & 33.349 & 0.006 &
0.334 \\
12 & 99.858 & 99.744 & 99.905 & 99.867 & 33.286 & 33.280 & 0.006 &
0.333 \\
13 & 99.980 & 99.878 & 99.915 & 99.958 & 33.308 & 33.306 & 0.003 &
0.333 \\
14 & 100.110 & 100.019 & 100.165 & 100.043 & 33.368 & 33.359 & 0.009 &
0.333 \\
15 & 100.095 & 100.124 & 100.090 & 100.110 & 33.376 & 33.369 & 0.007 &
0.333 \\
16 & 99.968 & 99.979 & 99.984 & 99.981 & 33.351 & 33.327 & 0.024 &
0.334 \\
17 & 100.057 & 100.078 & 100.166 & 100.171 & 33.373 & 33.379 & -0.007 &
0.334 \\
18 & 99.897 & 99.885 & 99.888 & 100.061 & 33.304 & 33.315 & -0.011 &
0.333 \\
19 & 99.856 & 99.790 & 99.856 & 99.870 & 33.281 & 33.280 & 0.002 &
0.333 \\
20 & 99.971 & 99.955 & 100.086 & 99.948 & 33.354 & 33.332 & 0.022 &
0.334 \\
22 & 99.895 & 99.792 & 99.833 & 99.839 & 33.289 & 33.274 & 0.015 &
0.333 \\
25 & 100.140 & 99.969 & 100.260 & 100.097 & 33.389 & 33.370 & 0.020 &
0.333 \\
26 & 99.920 & 99.968 & 99.884 & 99.923 & 33.307 & 33.308 & -0.002 &
0.333 \\
27 & 100.025 & 99.991 & 100.153 & 99.970 & 33.352 & 33.346 & 0.006 &
0.333 \\
28 & 100.000 & 99.940 & 100.028 & 100.013 & 33.351 & 33.331 & 0.020 &
0.334 \\
29 & 99.960 & 100.023 & 99.911 & 99.962 & 33.321 & 33.322 & -0.001 &
0.333 \\
30 & 99.999 & 100.052 & 100.067 & 99.986 & 33.340 & 33.345 & -0.005 &
0.333 \\
\bottomrule
\end{longtable}

\begin{longtable}[]{@{}rrrrrrrrr@{}}
\caption{三つのサンプルのどれかが独立でない場合}\tabularnewline
\toprule
No & 母集団 & 標本a & 標本b & 標本c & 加法1 & 加法2 & 差異 & 母集団比 \\
\midrule
\endfirsthead
\toprule
No & 母集団 & 標本a & 標本b & 標本c & 加法1 & 加法2 & 差異 & 母集団比 \\
\midrule
\endhead
2 & 99.968 & 100.051 & 99.987 & 100.012 & 33.292 & 33.339 & -0.047 &
0.333 \\
21 & 99.966 & 99.894 & 100.034 & 99.932 & 33.297 & 33.318 & -0.021 &
0.333 \\
23 & 100.047 & 99.977 & 99.869 & 100.056 & 33.289 & 33.322 & -0.034 &
0.333 \\
24 & 99.930 & 99.913 & 99.977 & 99.899 & 33.343 & 33.310 & 0.033 &
0.334 \\
\bottomrule
\end{longtable}

\[\mbox{加法1} = var(\frac{a + b + c}{3}), \mbox{加法2} = var(\frac{a}{3}) + var(\frac{b}{3}) + var(\frac{c}{3})\]

\newpage

\hypertarget{ux307eux3068ux3081}{%
\section{まとめ}\label{ux307eux3068ux3081}}

 データが独立であれば分散の加法性が成り立っており、\(n\)個の平均をとった場合、分散が\(\frac{1}{n}\)になることが予想できます。

 

\hypertarget{about-handout-style}{%
\section{About handout style}\label{about-handout-style}}

The Tufte handout style is a style that Edward Tufte uses in his books
and handouts. Tufte's style is known for its extensive use of sidenotes,
tight integration of graphics with text, and well-set typography. This
style has been implemented in LaTeX and HTML/CSS\footnote{See Github
  repositories
  \href{https://github.com/tufte-latex/tufte-latex}{tufte-latex} and
  \href{https://github.com/edwardtufte/tufte-css}{tufte-css}},
respectively.

 

\bibliography{bib/references.bib}



\end{document}

\documentclass[a4paper]{tufte-handout}

% ams
\usepackage{amssymb,amsmath}

\usepackage{ifxetex,ifluatex}
\usepackage{fixltx2e} % provides \textsubscript
\ifnum 0\ifxetex 1\fi\ifluatex 1\fi=0 % if pdftex
  \usepackage[T1]{fontenc}
  \usepackage[utf8]{inputenc}
\else % if luatex or xelatex
  \makeatletter
  \@ifpackageloaded{fontspec}{}{\usepackage{fontspec}}
  \makeatother
  \defaultfontfeatures{Ligatures=TeX,Scale=MatchLowercase}
  \makeatletter
  \@ifpackageloaded{soul}{
     \renewcommand\allcapsspacing[1]{{\addfontfeature{LetterSpace=15}#1}}
     \renewcommand\smallcapsspacing[1]{{\addfontfeature{LetterSpace=10}#1}}
   }{}
  \makeatother

\fi

% graphix
\usepackage{graphicx}
\setkeys{Gin}{width=\linewidth,totalheight=\textheight,keepaspectratio}

% booktabs
\usepackage{booktabs}

% url
\usepackage{url}

% hyperref
\usepackage{hyperref}

% units.
\usepackage{units}


\setcounter{secnumdepth}{-1}

% citations
\usepackage{natbib}
\bibliographystyle{plainnat}


% pandoc syntax highlighting
\usepackage{color}
\usepackage{fancyvrb}
\newcommand{\VerbBar}{|}
\newcommand{\VERB}{\Verb[commandchars=\\\{\}]}
\DefineVerbatimEnvironment{Highlighting}{Verbatim}{commandchars=\\\{\}}
% Add ',fontsize=\small' for more characters per line
\newenvironment{Shaded}{}{}
\newcommand{\AlertTok}[1]{\textcolor[rgb]{1.00,0.00,0.00}{\textbf{#1}}}
\newcommand{\AnnotationTok}[1]{\textcolor[rgb]{0.38,0.63,0.69}{\textbf{\textit{#1}}}}
\newcommand{\AttributeTok}[1]{\textcolor[rgb]{0.49,0.56,0.16}{#1}}
\newcommand{\BaseNTok}[1]{\textcolor[rgb]{0.25,0.63,0.44}{#1}}
\newcommand{\BuiltInTok}[1]{#1}
\newcommand{\CharTok}[1]{\textcolor[rgb]{0.25,0.44,0.63}{#1}}
\newcommand{\CommentTok}[1]{\textcolor[rgb]{0.38,0.63,0.69}{\textit{#1}}}
\newcommand{\CommentVarTok}[1]{\textcolor[rgb]{0.38,0.63,0.69}{\textbf{\textit{#1}}}}
\newcommand{\ConstantTok}[1]{\textcolor[rgb]{0.53,0.00,0.00}{#1}}
\newcommand{\ControlFlowTok}[1]{\textcolor[rgb]{0.00,0.44,0.13}{\textbf{#1}}}
\newcommand{\DataTypeTok}[1]{\textcolor[rgb]{0.56,0.13,0.00}{#1}}
\newcommand{\DecValTok}[1]{\textcolor[rgb]{0.25,0.63,0.44}{#1}}
\newcommand{\DocumentationTok}[1]{\textcolor[rgb]{0.73,0.13,0.13}{\textit{#1}}}
\newcommand{\ErrorTok}[1]{\textcolor[rgb]{1.00,0.00,0.00}{\textbf{#1}}}
\newcommand{\ExtensionTok}[1]{#1}
\newcommand{\FloatTok}[1]{\textcolor[rgb]{0.25,0.63,0.44}{#1}}
\newcommand{\FunctionTok}[1]{\textcolor[rgb]{0.02,0.16,0.49}{#1}}
\newcommand{\ImportTok}[1]{#1}
\newcommand{\InformationTok}[1]{\textcolor[rgb]{0.38,0.63,0.69}{\textbf{\textit{#1}}}}
\newcommand{\KeywordTok}[1]{\textcolor[rgb]{0.00,0.44,0.13}{\textbf{#1}}}
\newcommand{\NormalTok}[1]{#1}
\newcommand{\OperatorTok}[1]{\textcolor[rgb]{0.40,0.40,0.40}{#1}}
\newcommand{\OtherTok}[1]{\textcolor[rgb]{0.00,0.44,0.13}{#1}}
\newcommand{\PreprocessorTok}[1]{\textcolor[rgb]{0.74,0.48,0.00}{#1}}
\newcommand{\RegionMarkerTok}[1]{#1}
\newcommand{\SpecialCharTok}[1]{\textcolor[rgb]{0.25,0.44,0.63}{#1}}
\newcommand{\SpecialStringTok}[1]{\textcolor[rgb]{0.73,0.40,0.53}{#1}}
\newcommand{\StringTok}[1]{\textcolor[rgb]{0.25,0.44,0.63}{#1}}
\newcommand{\VariableTok}[1]{\textcolor[rgb]{0.10,0.09,0.49}{#1}}
\newcommand{\VerbatimStringTok}[1]{\textcolor[rgb]{0.25,0.44,0.63}{#1}}
\newcommand{\WarningTok}[1]{\textcolor[rgb]{0.38,0.63,0.69}{\textbf{\textit{#1}}}}

% longtable

% multiplecol
\usepackage{multicol}

% strikeout
\usepackage[normalem]{ulem}

% morefloats
\usepackage{morefloats}


% tightlist macro required by pandoc >= 1.14
\providecommand{\tightlist}{%
  \setlength{\itemsep}{0pt}\setlength{\parskip}{0pt}}

% title / author / date
\title[分散の加法性を視覚的に理解する(その3)]{分散の加法性を視覚的に理解する(その3)}
\author{Sampo Suzuki, CC 4.0 BY-NC-SA}
\date{2021-06-01}

% --- 参考資料 ----------------------------------------------------------------
% http://ctan.math.illinois.edu/language/japanese/zxjafont/zxjafont.pdf
% https://github.com/Gedevan-Aleksizde/Japan.R2019/blob/master/latex/preamble.tex
% https://teastat.blogspot.com/2019/01/bookdown.html

% --- Packages ----------------------------------------------------------------
% 日本語とtufte, kableExtraを使うために必要なTeXパッケージ指定
% \usepackage[pdfbox,tombo]{gentombow}    % トンボを設定する場合は有効にする
\usepackage{ifthen}                     % 条件分岐用 \ifthenelse{条件}{T}{F}
\usepackage{booktabs}                   % ここからkableExtra用パッケージ
\usepackage{longtable}                  % 
\usepackage{array}                      % 
\usepackage{multirow}                   % 
\usepackage{wrapfig}                    % 
\usepackage{float}                      % 
\usepackage{colortbl}                   % 
\usepackage{pdflscape}                  % 
\usepackage{tabu}                       % 
\usepackage{threeparttable}             % 
\usepackage{threeparttablex}            % 
\usepackage[normalem]{ulem}             % 
\usepackage{inputenc}                   % 
\usepackage{makecell}                   % 
\usepackage{xcolor}                     % ここまでkableExtra用
\usepackage{amsmath}                    % 
\usepackage{fontawesome5}               % fontawesomeを使うために必要
\usepackage{subfig}                     % 複数の図を並べる際に必要(古い?)
% \usepackage{subcaption}                 % 同上(新しい?)
\usepackage{zxjatype}                   % 日本語処理に必要
% \usepackage{xeCJK}                      % zxjatypeを読み込むと一緒に読み込まれる
\usepackage[noto]{zxjafont}             % Linux環境用
% \usepackage[haranoaji]{zxjafont}        % Windows環境用
% \usepackage[hiragino-pro]{zxjafont}     % macOS環境用(おそらく、駄目ならNotoで)
\usepackage{pxrubrica}                  % ルビ用
\usepackage{hyperref}                   % ハイパーリンク用必要?
\usepackage{booktabs}
\usepackage{longtable}
\usepackage{array}
\usepackage{multirow}
\usepackage{wrapfig}
\usepackage{float}
\usepackage{colortbl}
\usepackage{pdflscape}
\usepackage{tabu}
\usepackage{threeparttable}
\usepackage{threeparttablex}
\usepackage[normalem]{ulem}
\usepackage{makecell}
\usepackage{xcolor}

\begin{document}

\maketitle




\hypertarget{ux306fux3058ux3081ux306b}{%
\section{\texorpdfstring{\textbf{はじめに}}{はじめに}}\label{ux306fux3058ux3081ux306b}}

 分散の加法性を視覚的に理解する(その2)において、データが独立であれば分散の加法性がなりたつことがわかりました。では、同一正規分布から取り出した二つ、および、三つの値の平均値の場合はどうなるか、その2と同様の手段で確認してみます。

 

\hypertarget{ux540cux4e00ux30c7ux30fcux30bfux304bux3089ux30b5ux30f3ux30d7ux30eaux30f3ux30b0ux3057ux305fux4e8cux3064ux306eux5024ux3092ux5e73ux5747ux3057ux305fux5834ux5408}{%
\subsection{\texorpdfstring{\textbf{同一データからサンプリングした二つの値を平均した場合}}{同一データからサンプリングした二つの値を平均した場合}}\label{ux540cux4e00ux30c7ux30fcux30bfux304bux3089ux30b5ux30f3ux30d7ux30eaux30f3ux30b0ux3057ux305fux4e8cux3064ux306eux5024ux3092ux5e73ux5747ux3057ux305fux5834ux5408}}

 最初に以下の処理を行う関数を定義します。

\begin{itemize}
\tightlist
\item
  データを乱数生成する\footnote{今回は\texttt{rnorm()}関数による分散が\(100\)となる正規分布}
\item
  乱数生成したデータをランダムサンプリングする
\item
  作成したデータの統計量を求める
\item
  無相関検定の結果と統計量をデータフレームにまとめる
\end{itemize}

\begin{Shaded}
\begin{Highlighting}[numbers=left,,]
\NormalTok{f2 }\OtherTok{\textless{}{-}} \ControlFlowTok{function}\NormalTok{(}\AttributeTok{i =} \ConstantTok{NA}\NormalTok{, }\AttributeTok{n =} \DecValTok{5000000}\NormalTok{) \{}
  \CommentTok{\# データを乱数生成する}
\NormalTok{  x }\OtherTok{\textless{}{-}} \FunctionTok{rnorm}\NormalTok{(}\AttributeTok{n =}\NormalTok{ n, }\AttributeTok{mean =} \DecValTok{10}\NormalTok{, }\AttributeTok{sd =} \DecValTok{10}\NormalTok{)}
  \CommentTok{\# 乱数生成したデータから二つのデータを取り出す}
\NormalTok{  a }\OtherTok{\textless{}{-}} \FunctionTok{sample}\NormalTok{(x, n, }\AttributeTok{replace =} \ConstantTok{TRUE}\NormalTok{)}
\NormalTok{  b }\OtherTok{\textless{}{-}} \FunctionTok{sample}\NormalTok{(x, n, }\AttributeTok{replace =} \ConstantTok{TRUE}\NormalTok{)}
\NormalTok{  num }\OtherTok{\textless{}{-}} \DecValTok{2}
  \CommentTok{\# 統計量を求める}
\NormalTok{  df }\OtherTok{\textless{}{-}} \FunctionTok{data.frame}\NormalTok{(}\AttributeTok{no =}\NormalTok{ i,}
                   \AttributeTok{var.x =} \FunctionTok{var}\NormalTok{(x),}
                   \AttributeTok{var.a =} \FunctionTok{var}\NormalTok{(a), }\AttributeTok{var.b =} \FunctionTok{var}\NormalTok{(b),}
                   \AttributeTok{var.ab =} \FunctionTok{var}\NormalTok{((a }\SpecialCharTok{+}\NormalTok{ b) }\SpecialCharTok{/}\NormalTok{ num), }\AttributeTok{var.sum =}\NormalTok{ (}\FunctionTok{var}\NormalTok{(a }\SpecialCharTok{/}\NormalTok{ num) }\SpecialCharTok{+} \FunctionTok{var}\NormalTok{(b }\SpecialCharTok{/}\NormalTok{ num)),}
                   \AttributeTok{cov =} \FunctionTok{cov}\NormalTok{(a }\SpecialCharTok{/}\NormalTok{ num, b }\SpecialCharTok{/}\NormalTok{ num ),}
                   \AttributeTok{cov2 =} \FunctionTok{cov}\NormalTok{(a }\SpecialCharTok{/}\NormalTok{ num, b }\SpecialCharTok{/}\NormalTok{ num) }\SpecialCharTok{*} \DecValTok{2}\NormalTok{)}
  \CommentTok{\# 無相関の検定結果と統計量をデータフレームにまとめる}
\NormalTok{  df }\OtherTok{\textless{}{-}} \FunctionTok{cor.test}\NormalTok{(a, b) }\SpecialCharTok{\%\textgreater{}\%}\NormalTok{ broom}\SpecialCharTok{::}\FunctionTok{tidy}\NormalTok{() }\SpecialCharTok{\%\textgreater{}\%}\NormalTok{ dplyr}\SpecialCharTok{::}\FunctionTok{bind\_cols}\NormalTok{(df)}
  \FunctionTok{return}\NormalTok{(df)}
\NormalTok{\}}
\end{Highlighting}
\end{Shaded}

\newpage

\begin{table}

\caption{\label{tab:unnamed-chunk-3}二つのサンプルを平均した場合の分散}
\centering
\resizebox{\linewidth}{!}{
\begin{tabular}[t]{rrrrrrrrrrr}
\toprule
No & 相関係数 & p値 & 母集団 & 標本a & 標本b & 加法1 & 加法2 & 差異 & 母集団比 & cov2\\
\midrule
\cellcolor{gray!6}{1} & \cellcolor{gray!6}{-0.0004227} & \cellcolor{gray!6}{0.3446167} & \cellcolor{gray!6}{99.87213} & \cellcolor{gray!6}{99.87870} & \cellcolor{gray!6}{99.96647} & \cellcolor{gray!6}{49.94017} & \cellcolor{gray!6}{49.96129} & \cellcolor{gray!6}{-0.0211163} & \cellcolor{gray!6}{0.5000411} & \cellcolor{gray!6}{-0.0211163}\\
2 & 0.0006894 & 0.1231613 & 100.07524 & 99.96190 & 100.07439 & 50.04355 & 50.00907 & 0.0344784 & 0.5000593 & 0.0344784\\
\cellcolor{gray!6}{3} & \cellcolor{gray!6}{-0.0002985} & \cellcolor{gray!6}{0.5044151} & \cellcolor{gray!6}{100.03005} & \cellcolor{gray!6}{100.12848} & \cellcolor{gray!6}{100.12806} & \cellcolor{gray!6}{50.04919} & \cellcolor{gray!6}{50.06414} & \cellcolor{gray!6}{-0.0149462} & \cellcolor{gray!6}{0.5003415} & \cellcolor{gray!6}{-0.0149462}\\
4 & -0.0005679 & 0.2041367 & 100.05189 & 99.95022 & 100.02934 & 49.96650 & 49.99489 & -0.0283919 & 0.4994058 & -0.0283919\\
\cellcolor{gray!6}{5} & \cellcolor{gray!6}{-0.0007649} & \cellcolor{gray!6}{0.0871787} & \cellcolor{gray!6}{100.02386} & \cellcolor{gray!6}{99.96435} & \cellcolor{gray!6}{99.96240} & \cellcolor{gray!6}{49.94345} & \cellcolor{gray!6}{49.98169} & \cellcolor{gray!6}{-0.0382333} & \cellcolor{gray!6}{0.4993154} & \cellcolor{gray!6}{-0.0382333}\\
\addlinespace
6 & 0.0004128 & 0.3560166 & 99.96980 & 100.00515 & 100.05158 & 50.03483 & 50.01418 & 0.0206444 & 0.5004994 & 0.0206444\\
\cellcolor{gray!6}{7} & \cellcolor{gray!6}{-0.0003800} & \cellcolor{gray!6}{0.3955490} & \cellcolor{gray!6}{99.98493} & \cellcolor{gray!6}{99.96132} & \cellcolor{gray!6}{99.98555} & \cellcolor{gray!6}{49.96773} & \cellcolor{gray!6}{49.98672} & \cellcolor{gray!6}{-0.0189925} & \cellcolor{gray!6}{0.4997526} & \cellcolor{gray!6}{-0.0189925}\\
8 & -0.0004313 & 0.3348360 & 100.02496 & 100.04873 & 99.96994 & 49.98310 & 50.00467 & -0.0215671 & 0.4997063 & -0.0215671\\
\cellcolor{gray!6}{9} & \cellcolor{gray!6}{0.0001845} & \cellcolor{gray!6}{0.6798832} & \cellcolor{gray!6}{99.99130} & \cellcolor{gray!6}{99.91004} & \cellcolor{gray!6}{100.03260} & \cellcolor{gray!6}{49.99488} & \cellcolor{gray!6}{49.98566} & \cellcolor{gray!6}{0.0092239} & \cellcolor{gray!6}{0.4999924} & \cellcolor{gray!6}{0.0092239}\\
10 & -0.0006684 & 0.1350209 & 100.09096 & 100.13762 & 100.01318 & 50.00425 & 50.03770 & -0.0334453 & 0.4995881 & -0.0334453\\
\addlinespace
\cellcolor{gray!6}{11} & \cellcolor{gray!6}{0.0005206} & \cellcolor{gray!6}{0.2444083} & \cellcolor{gray!6}{99.92114} & \cellcolor{gray!6}{99.93640} & \cellcolor{gray!6}{99.94230} & \cellcolor{gray!6}{49.99569} & \cellcolor{gray!6}{49.96968} & \cellcolor{gray!6}{0.0260129} & \cellcolor{gray!6}{0.5003515} & \cellcolor{gray!6}{0.0260129}\\
12 & 0.0007688 & 0.0856032 & 100.03008 & 100.12376 & 99.95333 & 50.05773 & 50.01927 & 0.0384542 & 0.5004268 & 0.0384542\\
\cellcolor{gray!6}{13} & \cellcolor{gray!6}{-0.0001434} & \cellcolor{gray!6}{0.7484204} & \cellcolor{gray!6}{100.01911} & \cellcolor{gray!6}{99.99394} & \cellcolor{gray!6}{100.09906} & \cellcolor{gray!6}{50.01607} & \cellcolor{gray!6}{50.02325} & \cellcolor{gray!6}{-0.0071749} & \cellcolor{gray!6}{0.5000652} & \cellcolor{gray!6}{-0.0071749}\\
14 & 0.0000048 & 0.9914207 & 100.10705 & 100.24055 & 100.10889 & 50.08760 & 50.08736 & 0.0002409 & 0.5003404 & 0.0002409\\
\cellcolor{gray!6}{16} & \cellcolor{gray!6}{0.0001215} & \cellcolor{gray!6}{0.7857846} & \cellcolor{gray!6}{100.05458} & \cellcolor{gray!6}{100.13252} & \cellcolor{gray!6}{100.17808} & \cellcolor{gray!6}{50.08374} & \cellcolor{gray!6}{50.07765} & \cellcolor{gray!6}{0.0060868} & \cellcolor{gray!6}{0.5005642} & \cellcolor{gray!6}{0.0060868}\\
\addlinespace
17 & -0.0006408 & 0.1518697 & 99.85948 & 99.88638 & 99.86088 & 49.90481 & 49.93681 & -0.0320014 & 0.4997504 & -0.0320014\\
\cellcolor{gray!6}{18} & \cellcolor{gray!6}{0.0001450} & \cellcolor{gray!6}{0.7456971} & \cellcolor{gray!6}{100.02241} & \cellcolor{gray!6}{99.93323} & \cellcolor{gray!6}{99.93438} & \cellcolor{gray!6}{49.97415} & \cellcolor{gray!6}{49.96690} & \cellcolor{gray!6}{0.0072472} & \cellcolor{gray!6}{0.4996295} & \cellcolor{gray!6}{0.0072472}\\
19 & -0.0003088 & 0.4899390 & 100.12372 & 100.02042 & 100.19305 & 50.03791 & 50.05337 & -0.0154544 & 0.4997608 & -0.0154544\\
\cellcolor{gray!6}{20} & \cellcolor{gray!6}{-0.0001323} & \cellcolor{gray!6}{0.7673047} & \cellcolor{gray!6}{99.92560} & \cellcolor{gray!6}{99.96056} & \cellcolor{gray!6}{99.95999} & \cellcolor{gray!6}{49.97352} & \cellcolor{gray!6}{49.98014} & \cellcolor{gray!6}{-0.0066140} & \cellcolor{gray!6}{0.5001073} & \cellcolor{gray!6}{-0.0066140}\\
21 & -0.0002764 & 0.5365530 & 100.02100 & 100.03889 & 99.93493 & 49.97964 & 49.99345 & -0.0138179 & 0.4996915 & -0.0138179\\
\addlinespace
\cellcolor{gray!6}{22} & \cellcolor{gray!6}{0.0005443} & \cellcolor{gray!6}{0.2235309} & \cellcolor{gray!6}{100.07907} & \cellcolor{gray!6}{100.06634} & \cellcolor{gray!6}{100.10017} & \cellcolor{gray!6}{50.06887} & \cellcolor{gray!6}{50.04163} & \cellcolor{gray!6}{0.0272399} & \cellcolor{gray!6}{0.5002931} & \cellcolor{gray!6}{0.0272399}\\
23 & -0.0000553 & 0.9015291 & 99.92606 & 100.04581 & 99.91932 & 49.98852 & 49.99128 & -0.0027662 & 0.5002551 & -0.0027662\\
\cellcolor{gray!6}{24} & \cellcolor{gray!6}{-0.0006679} & \cellcolor{gray!6}{0.1353417} & \cellcolor{gray!6}{100.00815} & \cellcolor{gray!6}{99.94482} & \cellcolor{gray!6}{99.99480} & \cellcolor{gray!6}{49.95152} & \cellcolor{gray!6}{49.98491} & \cellcolor{gray!6}{-0.0333826} & \cellcolor{gray!6}{0.4994745} & \cellcolor{gray!6}{-0.0333826}\\
25 & 0.0005606 & 0.2100268 & 100.07681 & 99.99547 & 100.04771 & 50.03883 & 50.01080 & 0.0280350 & 0.5000043 & 0.0280350\\
\cellcolor{gray!6}{26} & \cellcolor{gray!6}{0.0001202} & \cellcolor{gray!6}{0.7881156} & \cellcolor{gray!6}{100.16421} & \cellcolor{gray!6}{100.13795} & \cellcolor{gray!6}{100.28389} & \cellcolor{gray!6}{50.11148} & \cellcolor{gray!6}{50.10546} & \cellcolor{gray!6}{0.0060223} & \cellcolor{gray!6}{0.5002933} & \cellcolor{gray!6}{0.0060223}\\
\addlinespace
27 & 0.0002696 & 0.5466506 & 100.00794 & 100.10217 & 100.04631 & 50.05061 & 50.03712 & 0.0134887 & 0.5004664 & 0.0134887\\
\cellcolor{gray!6}{28} & \cellcolor{gray!6}{0.0002009} & \cellcolor{gray!6}{0.6532036} & \cellcolor{gray!6}{99.84813} & \cellcolor{gray!6}{99.85202} & \cellcolor{gray!6}{99.85932} & \cellcolor{gray!6}{49.93787} & \cellcolor{gray!6}{49.92784} & \cellcolor{gray!6}{0.0100325} & \cellcolor{gray!6}{0.5001382} & \cellcolor{gray!6}{0.0100325}\\
29 & -0.0002756 & 0.5377413 & 99.94330 & 99.96889 & 99.89002 & 49.95096 & 49.96473 & -0.0137697 & 0.4997930 & -0.0137697\\
\cellcolor{gray!6}{31} & \cellcolor{gray!6}{-0.0002466} & \cellcolor{gray!6}{0.5813051} & \cellcolor{gray!6}{100.01894} & \cellcolor{gray!6}{100.03143} & \cellcolor{gray!6}{100.01167} & \cellcolor{gray!6}{49.99844} & \cellcolor{gray!6}{50.01078} & \cellcolor{gray!6}{-0.0123341} & \cellcolor{gray!6}{0.4998897} & \cellcolor{gray!6}{-0.0123341}\\
32 & -0.0006443 & 0.1496653 & 99.89803 & 99.91691 & 99.90149 & 49.92241 & 49.95460 & -0.0321861 & 0.4997337 & -0.0321861\\
\addlinespace
\cellcolor{gray!6}{33} & \cellcolor{gray!6}{-0.0001251} & \cellcolor{gray!6}{0.7797496} & \cellcolor{gray!6}{100.01071} & \cellcolor{gray!6}{99.98714} & \cellcolor{gray!6}{100.11296} & \cellcolor{gray!6}{50.01877} & \cellcolor{gray!6}{50.02502} & \cellcolor{gray!6}{-0.0062562} & \cellcolor{gray!6}{0.5001341} & \cellcolor{gray!6}{-0.0062562}\\
34 & -0.0000266 & 0.9525761 & 99.96254 & 99.89658 & 99.96925 & 49.96513 & 49.96646 & -0.0013289 & 0.4998385 & -0.0013289\\
\cellcolor{gray!6}{35} & \cellcolor{gray!6}{-0.0000921} & \cellcolor{gray!6}{0.8368399} & \cellcolor{gray!6}{99.92419} & \cellcolor{gray!6}{99.87316} & \cellcolor{gray!6}{99.96447} & \cellcolor{gray!6}{49.95481} & \cellcolor{gray!6}{49.95941} & \cellcolor{gray!6}{-0.0046012} & \cellcolor{gray!6}{0.4999271} & \cellcolor{gray!6}{-0.0046012}\\
36 & 0.0003492 & 0.4349171 & 100.01245 & 100.05928 & 99.96914 & 50.02457 & 50.00711 & 0.0174618 & 0.5001834 & 0.0174618\\
\cellcolor{gray!6}{38} & \cellcolor{gray!6}{-0.0006341} & \cellcolor{gray!6}{0.1562326} & \cellcolor{gray!6}{99.96110} & \cellcolor{gray!6}{100.07149} & \cellcolor{gray!6}{99.82457} & \cellcolor{gray!6}{49.94233} & \cellcolor{gray!6}{49.97401} & \cellcolor{gray!6}{-0.0316877} & \cellcolor{gray!6}{0.4996176} & \cellcolor{gray!6}{-0.0316877}\\
\addlinespace
39 & -0.0008329 & 0.0625534 & 99.94755 & 99.99783 & 99.95544 & 49.94668 & 49.98832 & -0.0416337 & 0.4997289 & -0.0416337\\
\cellcolor{gray!6}{40} & \cellcolor{gray!6}{0.0003846} & \cellcolor{gray!6}{0.3897481} & \cellcolor{gray!6}{100.00907} & \cellcolor{gray!6}{99.92099} & \cellcolor{gray!6}{100.02583} & \cellcolor{gray!6}{50.00593} & \cellcolor{gray!6}{49.98670} & \cellcolor{gray!6}{0.0192267} & \cellcolor{gray!6}{0.5000140} & \cellcolor{gray!6}{0.0192267}\\
41 & 0.0002062 & 0.6447010 & 99.95999 & 100.00154 & 100.01233 & 50.01378 & 50.00347 & 0.0103120 & 0.5003380 & 0.0103120\\
\cellcolor{gray!6}{42} & \cellcolor{gray!6}{0.0001978} & \cellcolor{gray!6}{0.6582189} & \cellcolor{gray!6}{100.08461} & \cellcolor{gray!6}{100.20002} & \cellcolor{gray!6}{100.07181} & \cellcolor{gray!6}{50.07786} & \cellcolor{gray!6}{50.06796} & \cellcolor{gray!6}{0.0099052} & \cellcolor{gray!6}{0.5003553} & \cellcolor{gray!6}{0.0099052}\\
44 & 0.0007321 & 0.1016436 & 100.01251 & 99.96788 & 100.08360 & 50.04948 & 50.01287 & 0.0366124 & 0.5004322 & 0.0366124\\
\addlinespace
\cellcolor{gray!6}{45} & \cellcolor{gray!6}{0.0004272} & \cellcolor{gray!6}{0.3394427} & \cellcolor{gray!6}{99.99145} & \cellcolor{gray!6}{100.02156} & \cellcolor{gray!6}{99.93629} & \cellcolor{gray!6}{50.01082} & \cellcolor{gray!6}{49.98946} & \cellcolor{gray!6}{0.0213559} & \cellcolor{gray!6}{0.5001509} & \cellcolor{gray!6}{0.0213559}\\
\bottomrule
\end{tabular}}
\end{table}

\begin{table}

\caption{\label{tab:unnamed-chunk-3}二つのサンプルが独立でない場合}
\centering
\resizebox{\linewidth}{!}{
\begin{tabular}[t]{rrrrrrrrrrr}
\toprule
No & 相関係数 & p値 & 母集団 & 標本a & 標本b & 加法1 & 加法2 & 差異 & 母集団比 & cov2\\
\midrule
\cellcolor{gray!6}{15} & \cellcolor{gray!6}{0.0010751} & \cellcolor{gray!6}{0.0162212} & \cellcolor{gray!6}{100.08494} & \cellcolor{gray!6}{100.1292} & \cellcolor{gray!6}{100.13671} & \cellcolor{gray!6}{50.12029} & \cellcolor{gray!6}{50.06647} & \cellcolor{gray!6}{0.0538243} & \cellcolor{gray!6}{0.5007776} & \cellcolor{gray!6}{0.0538243}\\
30 & 0.0009400 & 0.0355671 & 100.06054 & 100.1077 & 100.10518 & 50.10027 & 50.05322 & 0.0470486 & 0.5006996 & 0.0470486\\
\cellcolor{gray!6}{37} & \cellcolor{gray!6}{0.0009437} & \cellcolor{gray!6}{0.0348487} & \cellcolor{gray!6}{99.96085} & \cellcolor{gray!6}{100.0225} & \cellcolor{gray!6}{99.95927} & \cellcolor{gray!6}{50.04263} & \cellcolor{gray!6}{49.99545} & \cellcolor{gray!6}{0.0471792} & \cellcolor{gray!6}{0.5006223} & \cellcolor{gray!6}{0.0471792}\\
43 & 0.0011961 & 0.0074831 & 100.01955 & 100.0622 & 99.96043 & 50.06548 & 50.00566 & 0.0598113 & 0.5005569 & 0.0598113\\
\bottomrule
\end{tabular}}
\end{table}

\[\mbox{加法1} = var(\frac{a + b}{2}), \mbox{加法2} = var(\frac{a}{2}) + var(\frac{b}{2})\]

\newpage

\hypertarget{ux540cux4e00ux30c7ux30fcux30bfux304bux3089ux30b5ux30f3ux30d7ux30eaux30f3ux30b0ux3057ux305fux4e09ux3064ux306eux5024ux3092ux5e73ux5747ux3057ux305fux5834ux5408}{%
\subsection{\texorpdfstring{\textbf{同一データからサンプリングした三つの値を平均した場合}}{同一データからサンプリングした三つの値を平均した場合}}\label{ux540cux4e00ux30c7ux30fcux30bfux304bux3089ux30b5ux30f3ux30d7ux30eaux30f3ux30b0ux3057ux305fux4e09ux3064ux306eux5024ux3092ux5e73ux5747ux3057ux305fux5834ux5408}}

 最初に以下の処理を行う関数を定義します。

\begin{itemize}
\tightlist
\item
  データを乱数生成する\footnote{今回は\texttt{rnorm()}関数による分散が\(100\)となる正規分布}
\item
  乱数生成したデータをランダムサンプリングする
\item
  作成したデータの統計量を求める
\item
  無相関検定の結果と統計量をデータフレームにまとめる
\end{itemize}

\begin{Shaded}
\begin{Highlighting}[numbers=left,,]
\NormalTok{f3 }\OtherTok{\textless{}{-}} \ControlFlowTok{function}\NormalTok{(}\AttributeTok{i =} \ConstantTok{NA}\NormalTok{, }\AttributeTok{n =} \DecValTok{5000000}\NormalTok{) \{}
  \CommentTok{\# データを乱数生成する}
\NormalTok{  x }\OtherTok{\textless{}{-}} \FunctionTok{rnorm}\NormalTok{(}\AttributeTok{n =}\NormalTok{ n, }\AttributeTok{mean =} \DecValTok{10}\NormalTok{, }\AttributeTok{sd =} \DecValTok{10}\NormalTok{)}
  \CommentTok{\# 乱数生成したデータから三つのデータを取り出す}
\NormalTok{  a }\OtherTok{\textless{}{-}} \FunctionTok{sample}\NormalTok{(x, n, }\AttributeTok{replace =} \ConstantTok{TRUE}\NormalTok{)}
\NormalTok{  b }\OtherTok{\textless{}{-}} \FunctionTok{sample}\NormalTok{(x, n, }\AttributeTok{replace =} \ConstantTok{TRUE}\NormalTok{)}
\NormalTok{  c }\OtherTok{\textless{}{-}} \FunctionTok{sample}\NormalTok{(x, n, }\AttributeTok{replace =} \ConstantTok{TRUE}\NormalTok{)}
\NormalTok{  num }\OtherTok{\textless{}{-}} \DecValTok{3}
  \CommentTok{\# 統計量を求める}
\NormalTok{  df }\OtherTok{\textless{}{-}} \FunctionTok{data.frame}\NormalTok{(}\AttributeTok{no =}\NormalTok{ i,}
                   \AttributeTok{var.x =} \FunctionTok{var}\NormalTok{(x), }
                   \AttributeTok{var.a =} \FunctionTok{var}\NormalTok{(a), }\AttributeTok{var.b =} \FunctionTok{var}\NormalTok{(b), }\AttributeTok{var.c =} \FunctionTok{var}\NormalTok{(c),}
                   \AttributeTok{var.abc =} \FunctionTok{var}\NormalTok{((a }\SpecialCharTok{+}\NormalTok{ b }\SpecialCharTok{+}\NormalTok{ c) }\SpecialCharTok{/}\NormalTok{ num),}
                   \AttributeTok{var.sum =}\NormalTok{ (}\FunctionTok{var}\NormalTok{(a }\SpecialCharTok{/}\NormalTok{ num) }\SpecialCharTok{+} \FunctionTok{var}\NormalTok{(b }\SpecialCharTok{/}\NormalTok{ num) }\SpecialCharTok{+} \FunctionTok{var}\NormalTok{(c }\SpecialCharTok{/}\NormalTok{ num)),}
                   \AttributeTok{cov.ab =} \FunctionTok{cov}\NormalTok{(a, b), }\AttributeTok{cov.ac =} \FunctionTok{cov}\NormalTok{(a, c), }\AttributeTok{cov.bc =} \FunctionTok{cov}\NormalTok{(b, c),}
                   \AttributeTok{cov2.ab =} \FunctionTok{cov}\NormalTok{(a, b) }\SpecialCharTok{*} \DecValTok{2}\NormalTok{, }\AttributeTok{cov2.ac =} \FunctionTok{cov}\NormalTok{(a, c) }\SpecialCharTok{*} \DecValTok{2}\NormalTok{, }\AttributeTok{cov2.bc =} \FunctionTok{cov}\NormalTok{(b, c) }\SpecialCharTok{*} \DecValTok{2}\NormalTok{)}
  \CommentTok{\# 無相関の検定結果と統計量をデータフレームにまとめる}
\NormalTok{  df }\OtherTok{\textless{}{-}} \FunctionTok{cor.test}\NormalTok{(a, b) }\SpecialCharTok{\%\textgreater{}\%}\NormalTok{ broom}\SpecialCharTok{::}\FunctionTok{tidy}\NormalTok{() }\SpecialCharTok{\%\textgreater{}\%}\NormalTok{ dplyr}\SpecialCharTok{::}\FunctionTok{bind\_cols}\NormalTok{(df)}
\NormalTok{  df }\OtherTok{\textless{}{-}} \FunctionTok{cor.test}\NormalTok{(a, c) }\SpecialCharTok{\%\textgreater{}\%}\NormalTok{ broom}\SpecialCharTok{::}\FunctionTok{tidy}\NormalTok{() }\SpecialCharTok{\%\textgreater{}\%}\NormalTok{ dplyr}\SpecialCharTok{::}\FunctionTok{bind\_cols}\NormalTok{(df)}
\NormalTok{  df }\OtherTok{\textless{}{-}} \FunctionTok{cor.test}\NormalTok{(b, c) }\SpecialCharTok{\%\textgreater{}\%}\NormalTok{ broom}\SpecialCharTok{::}\FunctionTok{tidy}\NormalTok{() }\SpecialCharTok{\%\textgreater{}\%}\NormalTok{ dplyr}\SpecialCharTok{::}\FunctionTok{bind\_cols}\NormalTok{(df)}
  \FunctionTok{return}\NormalTok{(df)}
\NormalTok{\}}
\end{Highlighting}
\end{Shaded}

\newpage

\begin{table}

\caption{\label{tab:unnamed-chunk-6}三つのサンプルを平均した場合の分散}
\centering
\resizebox{\linewidth}{!}{
\begin{tabular}[t]{rrrrrrrrr}
\toprule
No & 母集団 & 標本a & 標本b & 標本c & 加法1 & 加法2 & 差異 & 母集団比\\
\midrule
\cellcolor{gray!6}{1} & \cellcolor{gray!6}{100.04444} & \cellcolor{gray!6}{99.99364} & \cellcolor{gray!6}{99.95237} & \cellcolor{gray!6}{99.97451} & \cellcolor{gray!6}{33.30230} & \cellcolor{gray!6}{33.32450} & \cellcolor{gray!6}{-0.0221988} & \cellcolor{gray!6}{0.3328751}\\
3 & 99.97346 & 99.93094 & 99.92829 & 99.87890 & 33.30009 & 33.30424 & -0.0041419 & 0.3330893\\
\cellcolor{gray!6}{4} & \cellcolor{gray!6}{100.04684} & \cellcolor{gray!6}{99.99903} & \cellcolor{gray!6}{100.01119} & \cellcolor{gray!6}{100.09931} & \cellcolor{gray!6}{33.32948} & \cellcolor{gray!6}{33.34550} & \cellcolor{gray!6}{-0.0160241} & \cellcolor{gray!6}{0.3331388}\\
5 & 99.98713 & 100.06519 & 99.94869 & 99.98973 & 33.30402 & 33.33373 & -0.0297117 & 0.3330831\\
\cellcolor{gray!6}{6} & \cellcolor{gray!6}{100.11323} & \cellcolor{gray!6}{100.05457} & \cellcolor{gray!6}{100.20764} & \cellcolor{gray!6}{100.13805} & \cellcolor{gray!6}{33.38514} & \cellcolor{gray!6}{33.37781} & \cellcolor{gray!6}{0.0073303} & \cellcolor{gray!6}{0.3334738}\\
\addlinespace
7 & 99.97585 & 100.06765 & 99.98634 & 99.93075 & 33.29689 & 33.33164 & -0.0347474 & 0.3330493\\
\cellcolor{gray!6}{8} & \cellcolor{gray!6}{100.03576} & \cellcolor{gray!6}{100.04278} & \cellcolor{gray!6}{100.07724} & \cellcolor{gray!6}{100.04879} & \cellcolor{gray!6}{33.36852} & \cellcolor{gray!6}{33.35209} & \cellcolor{gray!6}{0.0164251} & \cellcolor{gray!6}{0.3335659}\\
10 & 100.02147 & 100.00110 & 100.02245 & 99.87074 & 33.36918 & 33.32159 & 0.0475916 & 0.3336202\\
\cellcolor{gray!6}{12} & \cellcolor{gray!6}{100.05354} & \cellcolor{gray!6}{99.96653} & \cellcolor{gray!6}{100.00867} & \cellcolor{gray!6}{100.04385} & \cellcolor{gray!6}{33.29563} & \cellcolor{gray!6}{33.33545} & \cellcolor{gray!6}{-0.0398189} & \cellcolor{gray!6}{0.3327781}\\
13 & 100.00835 & 99.95240 & 100.05820 & 100.07131 & 33.34059 & 33.34243 & -0.0018450 & 0.3333781\\
\addlinespace
\cellcolor{gray!6}{14} & \cellcolor{gray!6}{100.01183} & \cellcolor{gray!6}{100.01245} & \cellcolor{gray!6}{100.00605} & \cellcolor{gray!6}{99.98231} & \cellcolor{gray!6}{33.31892} & \cellcolor{gray!6}{33.33342} & \cellcolor{gray!6}{-0.0145007} & \cellcolor{gray!6}{0.3331498}\\
15 & 100.09281 & 100.10893 & 100.09232 & 99.98592 & 33.36939 & 33.35413 & 0.0152559 & 0.3333844\\
\cellcolor{gray!6}{16} & \cellcolor{gray!6}{99.98783} & \cellcolor{gray!6}{99.87088} & \cellcolor{gray!6}{99.97196} & \cellcolor{gray!6}{99.98309} & \cellcolor{gray!6}{33.30840} & \cellcolor{gray!6}{33.31399} & \cellcolor{gray!6}{-0.0055924} & \cellcolor{gray!6}{0.3331245}\\
17 & 99.95895 & 100.02779 & 100.00703 & 99.96990 & 33.33737 & 33.33386 & 0.0035097 & 0.3335106\\
\cellcolor{gray!6}{18} & \cellcolor{gray!6}{99.97938} & \cellcolor{gray!6}{99.92114} & \cellcolor{gray!6}{99.96560} & \cellcolor{gray!6}{99.97708} & \cellcolor{gray!6}{33.30683} & \cellcolor{gray!6}{33.31820} & \cellcolor{gray!6}{-0.0113705} & \cellcolor{gray!6}{0.3331370}\\
\addlinespace
19 & 100.06790 & 100.00039 & 99.97305 & 100.11321 & 33.35061 & 33.34296 & 0.0076534 & 0.3332799\\
\cellcolor{gray!6}{20} & \cellcolor{gray!6}{100.10186} & \cellcolor{gray!6}{100.05983} & \cellcolor{gray!6}{100.10586} & \cellcolor{gray!6}{100.10155} & \cellcolor{gray!6}{33.37596} & \cellcolor{gray!6}{33.36303} & \cellcolor{gray!6}{0.0129313} & \cellcolor{gray!6}{0.3334200}\\
21 & 99.97606 & 99.92166 & 99.93785 & 99.97295 & 33.30252 & 33.31472 & -0.0122004 & 0.3331049\\
\cellcolor{gray!6}{22} & \cellcolor{gray!6}{100.06831} & \cellcolor{gray!6}{100.04468} & \cellcolor{gray!6}{100.06287} & \cellcolor{gray!6}{99.99709} & \cellcolor{gray!6}{33.34573} & \cellcolor{gray!6}{33.34496} & \cellcolor{gray!6}{0.0007739} & \cellcolor{gray!6}{0.3332297}\\
23 & 99.84294 & 99.90149 & 99.71609 & 100.01128 & 33.28312 & 33.29210 & -0.0089743 & 0.3333548\\
\addlinespace
\cellcolor{gray!6}{24} & \cellcolor{gray!6}{100.06130} & \cellcolor{gray!6}{100.12195} & \cellcolor{gray!6}{100.07985} & \cellcolor{gray!6}{100.11951} & \cellcolor{gray!6}{33.33344} & \cellcolor{gray!6}{33.36903} & \cellcolor{gray!6}{-0.0355897} & \cellcolor{gray!6}{0.3331302}\\
25 & 100.09145 & 100.01542 & 100.12195 & 100.19300 & 33.37587 & 33.37004 & 0.0058327 & 0.3334538\\
\cellcolor{gray!6}{26} & \cellcolor{gray!6}{99.94882} & \cellcolor{gray!6}{99.96522} & \cellcolor{gray!6}{99.98467} & \cellcolor{gray!6}{99.90111} & \cellcolor{gray!6}{33.31360} & \cellcolor{gray!6}{33.31678} & \cellcolor{gray!6}{-0.0031763} & \cellcolor{gray!6}{0.3333066}\\
27 & 99.93525 & 99.99087 & 99.90116 & 99.96036 & 33.30939 & 33.31693 & -0.0075442 & 0.3333097\\
\cellcolor{gray!6}{28} & \cellcolor{gray!6}{100.03707} & \cellcolor{gray!6}{100.06852} & \cellcolor{gray!6}{99.91367} & \cellcolor{gray!6}{99.96827} & \cellcolor{gray!6}{33.32400} & \cellcolor{gray!6}{33.32783} & \cellcolor{gray!6}{-0.0038278} & \cellcolor{gray!6}{0.3331165}\\
\addlinespace
29 & 100.03579 & 99.96155 & 99.96476 & 100.06081 & 33.34713 & 33.33190 & 0.0152284 & 0.3333520\\
\cellcolor{gray!6}{30} & \cellcolor{gray!6}{100.10607} & \cellcolor{gray!6}{100.21994} & \cellcolor{gray!6}{100.18056} & \cellcolor{gray!6}{99.94590} & \cellcolor{gray!6}{33.37703} & \cellcolor{gray!6}{33.37182} & \cellcolor{gray!6}{0.0052115} & \cellcolor{gray!6}{0.3334167}\\
31 & 100.06611 & 100.07600 & 100.02569 & 100.00403 & 33.34717 & 33.34508 & 0.0020945 & 0.3332514\\
\cellcolor{gray!6}{32} & \cellcolor{gray!6}{100.02972} & \cellcolor{gray!6}{100.02388} & \cellcolor{gray!6}{100.01994} & \cellcolor{gray!6}{99.93302} & \cellcolor{gray!6}{33.31065} & \cellcolor{gray!6}{33.33076} & \cellcolor{gray!6}{-0.0201115} & \cellcolor{gray!6}{0.3330075}\\
33 & 99.93844 & 100.04248 & 99.93554 & 100.02755 & 33.31410 & 33.33395 & -0.0198471 & 0.3333463\\
\addlinespace
\cellcolor{gray!6}{34} & \cellcolor{gray!6}{100.12003} & \cellcolor{gray!6}{99.99567} & \cellcolor{gray!6}{100.06442} & \cellcolor{gray!6}{100.11670} & \cellcolor{gray!6}{33.33656} & \cellcolor{gray!6}{33.35298} & \cellcolor{gray!6}{-0.0164191} & \cellcolor{gray!6}{0.3329659}\\
35 & 99.97455 & 99.97492 & 100.06277 & 99.98965 & 33.30549 & 33.33637 & -0.0308814 & 0.3331397\\
\cellcolor{gray!6}{36} & \cellcolor{gray!6}{100.08121} & \cellcolor{gray!6}{100.14637} & \cellcolor{gray!6}{99.95326} & \cellcolor{gray!6}{100.18183} & \cellcolor{gray!6}{33.37774} & \cellcolor{gray!6}{33.36461} & \cellcolor{gray!6}{0.0131284} & \cellcolor{gray!6}{0.3335065}\\
37 & 99.94658 & 99.81552 & 100.10360 & 100.01504 & 33.33012 & 33.32602 & 0.0041007 & 0.3334793\\
\cellcolor{gray!6}{38} & \cellcolor{gray!6}{100.04696} & \cellcolor{gray!6}{100.11784} & \cellcolor{gray!6}{100.02709} & \cellcolor{gray!6}{100.12850} & \cellcolor{gray!6}{33.36266} & \cellcolor{gray!6}{33.36372} & \cellcolor{gray!6}{-0.0010547} & \cellcolor{gray!6}{0.3334700}\\
\addlinespace
39 & 100.08929 & 100.13179 & 100.18596 & 100.03494 & 33.38592 & 33.37252 & 0.0133982 & 0.3335614\\
\cellcolor{gray!6}{40} & \cellcolor{gray!6}{100.04294} & \cellcolor{gray!6}{99.96954} & \cellcolor{gray!6}{99.94056} & \cellcolor{gray!6}{100.08237} & \cellcolor{gray!6}{33.32970} & \cellcolor{gray!6}{33.33250} & \cellcolor{gray!6}{-0.0027918} & \cellcolor{gray!6}{0.3331540}\\
41 & 99.97840 & 100.00749 & 100.02821 & 100.00197 & 33.34887 & 33.33752 & 0.0113495 & 0.3335607\\
\cellcolor{gray!6}{42} & \cellcolor{gray!6}{99.96526} & \cellcolor{gray!6}{99.92862} & \cellcolor{gray!6}{99.92791} & \cellcolor{gray!6}{99.98663} & \cellcolor{gray!6}{33.31896} & \cellcolor{gray!6}{33.31591} & \cellcolor{gray!6}{0.0030531} & \cellcolor{gray!6}{0.3333054}\\
43 & 99.94242 & 99.94334 & 99.92731 & 99.84776 & 33.30515 & 33.30205 & 0.0031072 & 0.3332434\\
\addlinespace
\cellcolor{gray!6}{44} & \cellcolor{gray!6}{99.96083} & \cellcolor{gray!6}{99.92636} & \cellcolor{gray!6}{100.01802} & \cellcolor{gray!6}{100.00641} & \cellcolor{gray!6}{33.31566} & \cellcolor{gray!6}{33.32787} & \cellcolor{gray!6}{-0.0122018} & \cellcolor{gray!6}{0.3332872}\\
\bottomrule
\end{tabular}}
\end{table}

\begin{table}

\caption{\label{tab:unnamed-chunk-6}三つのサンプルのどれかが独立でない場合}
\centering
\resizebox{\linewidth}{!}{
\begin{tabular}[t]{rrrrrrrrr}
\toprule
No & 母集団 & 標本a & 標本b & 標本c & 加法1 & 加法2 & 差異 & 母集団比\\
\midrule
\cellcolor{gray!6}{2} & \cellcolor{gray!6}{100.01114} & \cellcolor{gray!6}{100.10941} & \cellcolor{gray!6}{100.10965} & \cellcolor{gray!6}{99.98592} & \cellcolor{gray!6}{33.35917} & \cellcolor{gray!6}{33.35611} & \cellcolor{gray!6}{0.0030629} & \cellcolor{gray!6}{0.3335546}\\
9 & 99.88774 & 99.86397 & 99.84461 & 99.92135 & 33.28043 & 33.29221 & -0.0117852 & 0.3331783\\
\cellcolor{gray!6}{11} & \cellcolor{gray!6}{100.10944} & \cellcolor{gray!6}{100.05398} & \cellcolor{gray!6}{100.12787} & \cellcolor{gray!6}{100.11281} & \cellcolor{gray!6}{33.37623} & \cellcolor{gray!6}{33.36607} & \cellcolor{gray!6}{0.0101604} & \cellcolor{gray!6}{0.3333975}\\
45 & 99.89511 & 99.76852 & 99.83532 & 99.81607 & 33.28638 & 33.26888 & 0.0175006 & 0.3332133\\
\bottomrule
\end{tabular}}
\end{table}

\[\mbox{加法1} = var(\frac{a + b + c}{3}), \mbox{加法2} = var(\frac{a}{3}) + var(\frac{b}{3}) + var(\frac{c}{3})\]

\newpage

\hypertarget{ux307eux3068ux3081}{%
\section{まとめ}\label{ux307eux3068ux3081}}

 データが独立であれば分散の加法性が成り立っており、\(n\)個の平均をとった場合、分散が\(\frac{1}{n}\)になることが予想できます。

 

\hypertarget{about-handout-style}{%
\section{About handout style}\label{about-handout-style}}

The Tufte handout style is a style that Edward Tufte uses in his books
and handouts. Tufte's style is known for its extensive use of sidenotes,
tight integration of graphics with text, and well-set typography. This
style has been implemented in LaTeX and HTML/CSS\footnote{See Github
  repositories
  \href{https://github.com/tufte-latex/tufte-latex}{tufte-latex} and
  \href{https://github.com/edwardtufte/tufte-css}{tufte-css}},
respectively.

 

\bibliography{bib/references.bib}



\end{document}
